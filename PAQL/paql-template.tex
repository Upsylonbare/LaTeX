%-----------------------------------
% Plan de test simplifié
%
% ProSE
%
% Auteurs : Thomas CRAVIC
%
%-----------------------------------

\documentclass[a4paper,11pt,titlepage]{article}

\usepackage[english,francais]{babel}
\usepackage[T1]{fontenc}
\usepackage[utf8]{inputenc}

\usepackage{xspace, graphicx}
\usepackage{amsmath}
\usepackage{amsfonts}
\usepackage{amssymb}
\usepackage{lastpage}
\usepackage{array}
\usepackage{tabularx}
\usepackage{hyperref}

\usepackage{titlesec}
\usepackage{float}
\usepackage[table,xcdraw]{xcolor}
\usepackage{color, soul}

%------------------------------------
% Variables
\newcommand{\version}{0.0}
\newcommand{\revision}{0}
\newcommand{\projectName}{Nom du Projet}
\newcommand{\documentName}{Nom du document}
\newcommand{\creator}{Clément Le Goffic}
\newcommand{\creatorAbrev}{C. Le Goffic}
\newcommand{\documentNameAbrev}{XXXX}
\newcommand{\annee}{2022}
\newcommand{\teamName}{Nom d'équipe}
\newcommand{\prose}{ProSE}
\newcommand{\teamNumber}{A?}
\newcommand{\completer}{\textbf{A compléter}}

%------------------------------------

\usepackage{fancyhdr}

\pagestyle{fancy}

\graphicspath{{./figs/}}

\setlength{\hoffset}{-40pt}

\setlength{\topmargin}{-25pt}
\setlength{\headsep}{10pt}

\renewcommand{\headheight}{80pt}

\renewcommand{\headwidth}{450pt}
\setlength{\textwidth}{450pt}
\setlength{\textheight}{604pt}

\renewcommand{\footrulewidth}{0.1mm}

\fancyhf{}
        \fancyhead[LO]{\bf \includegraphics[width=80pt]{eseo.png}\\
        						\medskip
                                 {\prose} équipe {\teamNumber} {\annee}}
         \fancyhead[RO]{\bf \includegraphics[width=40pt]{st.png}\\
				\medskip
				{\small{Ref. {\documentNameAbrev}\_E{\teamNumber}}}}
         \fancyfoot[LO]{\sl {\it Version {\version} - Révision {\revision}}}
         \cfoot{\copyright {\annee} {\teamName}}
         \fancyfoot[RO]{\thepage/{\completer}}

\setcounter{tocdepth}{3}
%%%%%%%%%%%%%%%%%%%%%%%%%%%%%%%%%%%%%%%%%%%%%%%%%%%%%%
\titleclass{\subsubsubsection}{straight}[\subsection]

\newcounter{subsubsubsection}[subsubsection]
\renewcommand\thesubsubsubsection{\thesubsubsection.\arabic{subsubsubsection}}
\renewcommand\theparagraph{\thesubsubsubsection.\arabic{paragraph}} % optional; useful if paragraphs are to be numbered

\titleformat{\subsubsubsection}
  {\normalfont\normalsize\bfseries}{\thesubsubsubsection}{1em}{}
\titlespacing*{\subsubsubsection}
{0pt}{3.25ex plus 1ex minus .2ex}{1.5ex plus .2ex}

\makeatletter
\renewcommand\paragraph{\@startsection{paragraph}{5}{\z@}%
  {3.25ex \@plus1ex \@minus.2ex}%
  {-1em}%
  {\normalfont\normalsize\bfseries}}
\renewcommand\subparagraph{\@startsection{subparagraph}{6}{\parindent}%
  {3.25ex \@plus1ex \@minus .2ex}%
  {-1em}%
  {\normalfont\normalsize\bfseries}}
\def\toclevel@subsubsubsection{4}
\def\toclevel@paragraph{5}
\def\toclevel@paragraph{6}
\def\l@subsubsubsection{\@dottedtocline{4}{7em}{4em}}
\def\l@paragraph{\@dottedtocline{5}{10em}{5em}}
\def\l@subparagraph{\@dottedtocline{6}{14em}{6em}}
\makeatother

\setcounter{secnumdepth}{4}
\setcounter{tocdepth}{4}
%%%%%%%%%%%%%%%%%%%%%%%%%%%%%%%%%%%%%%%%%%%%%
\newcommand{\tabitem}{~~\llap{\textbullet}~~}

%----------------------------------------
%       DOCUMENT
%----------------------------------------

\begin{document}

%---------------------------------------

\sloppy%
% Modifie l’espacement vertical entre les lignes d’un tableau (tabular)
\renewcommand{\arraystretch}{1.5}

%---------------------------------------

%\vspace{-2cm}%
\begin{center}%
\vspace*{1cm}
\rule[0.5ex]{0.4\textwidth}{0.1mm}\\
\vspace*{2mm}
{\Huge {\textsc{\bf {\documentName}}}}
\vspace{0.4cm}\\
{\large\bf {\prose} {\teamNumber} {\annee} - {\teamName}}\\
{\large\bf {\projectName}}

\rule[0.5ex]{0.4\textwidth}{0.1mm}

\vspace{1cm}

\end{center}
\begin{center}
\begin{tabular}{|c|c|}
\hline
Responsable du document & Xxxxxx Xxxxxx                \\
État du document        & En cours de création         \\
Version                 & {\version}  \\
Révision                & {\revision} \\
\hline
\end{tabular}
\end{center}

\vspace{2cm}
\noindent
\textbf{AVERTISSEMENT :}\\
Le présent document est un document à but pédagogique. 
Il a été réalisé sous la direction de Jérôme Delatour, 
en collaboration avec des enseignants 
et des étudiants de l'option SE du groupe ESEO. 
Ce document a été traduit en \LaTeX par Clément Le Goffic.\\
Ce document est la propriété de Jérôme Delatour du groupe ESEO. 
En dehors des activités pédagogiques de l'ESEO, ce document ne peut 
être diffusé ou recopié sans l’autorisation écrite de son propriétaire (Jérôme Delatour). %Comment if not needed

\newpage

% Auteur : Camille Constant

\noindent
% Modifie l’espacement horizontal entre les colonnes
%\setlength{\tabcolsep}{5pt}
\begin{tabularx}{\linewidth}{|p{1.8cm}|X|p{2cm}|p{1.5cm}|p{1.5cm}|}
\hline
\textbf{Date} & \textbf{Actions} & \textbf{Auteur} & \textbf{Version} & \textbf{Révision}  \\
\hline
11/02/2022 & Création du document & {\creatorAbrev} & 0.0 & 0\\
11/02/2022 & Appropriation du document & {\creatorAbrev} & 0.0 & 1\\
\hline
\end{tabularx}

\newpage

%------------------------------- 

\tableofcontents
% alternative pour réduire l'espacement entre les entrées de la table des matières
% (la valeur numérique peut être adaptée au besoin) : 
%{\setlength{\baselineskip}{0.96\baselineskip}\tableofcontents\par}
\newpage

%-------------------------------
% Ajouter toutes les parties les unes après les autres, séparées par un \newpage
% Exemple : 
% \input{Introduction}
% \newpage
\section{But}
\subsection{Objectifs du document}
Ce document est un Plan d’Assurance Qualité Logicielle (PAQL) visant à définir toutes les règles, 
les méthodes et les outils utilisés dans le projet {\projectName} afin de définir et 
contrôler la qualité du projet.\\
Ce document poursuit les objectifs suivants :
\begin{itemize}
  \item Définir le niveau de qualité attendu par l’équipe projet pour le projet ProSE.
  \item Définir les outils utilisés, les processus et procédures à suivre par l’équipe 
  projet tant au niveau organisationnel que technique lors du projet ProSE.
\end{itemize}
Ce document est disponible sur le Référentiel Documentaire Projet(RDP) dans le répertoire 
Qualité\slash Redaction\_PAQL sous le nom "{\documentName}".

\subsection{Portée}
Ce document est destiné à :
\begin{itemize}
  \item À l'équipe projet,
  \item Aux consultants de la société FORMATO
\end{itemize}

\subsection{Copyright}
Le présen document est la propriété de \completer

\subsection{Vue d’ensemble}
Ce PAQL est structuré suivant les grandes parties proposées par la norme [IEEE\-730\_1998].\\
Il est donc décomposé en 24 parties.\\
La norme IEEE 730 décrit le contenu d'un plan d'AQL pour un logiciel:
\begin{itemize}
  \item Intention et Portée
  \item Définitions et Abréviations
  \item Documents de références
  \item Survol du plan d'assurance qualité Logicielle
  \begin{itemize}
    \item Organisation
    \item Niveau de criticité du logiciel
    \item Outils, techniques et méthodologies
    \item Ressources
    \item Normes, pratiques et conventions
    \item Calendriers
  \end{itemize}
  \item Activités et tâches de cycle de vis de l'AQL
  \begin{itemize}
    \item Rôle de l'assurance de produit
    \item Rôle de l'assurance du processus
    \item Assurances sur les activités et les tâches du système de management de la qualité
    \item Activités et tâches additionnelles
  \end{itemize}
  \item Processus et politiques additionnelles
  \begin{itemize}
    \item Processus de revue de contrat
    \item Processus de mesure de la qualité
    \item Politiques de tests
    \item Politique de dérogation et de déviation
    \item Politique d'itération des tâches
  \end{itemize}
  \item Enregistrements et rapports de l'AQL
  \begin{itemize}
    \item Enregistrements
    \item Rapports
  \end{itemize}
\end{itemize}

\subsection{Références}

\nocite{*}
\bibliographystyle{plain}
\bibliography{biblio.bib}

\section{Gestion}
Cette partie décrit l’organisation, les tâches et les responsabilités 
en rapport avec les activités d’Assurance Qualité (AQ) du projet {\projectName}.

\subsection{Organisation}

\subsubsection{Projet concerné}
Le projet \completer


\subsubsection{Ressources humaines}

\subsubsubsection{Équipe Projet}
\noindent
% Modifie l’espacement horizontal entre les colonnes
%\setlength{\tabcolsep}{5pt}
\begin{tabularx}{\linewidth}{|X|X|X|p{5cm}|X|}
\hline
\textbf{Rôle} & \textbf{Nom} & \textbf{Prénom} & \textbf{Mail} & \textbf{Téléphone}  \\
\hline
Responsable Qualité Test & Le Goffic & Clément & clement.legoffic@reseau.eseo.fr & +33 7 82 77 51 25\\
\hline
\end{tabularx}

\subsubsubsection{Client}
\completer

\subsubsubsection{Consultants et auditeurs}
Si nécessaire, l'équipe projet pourra faire appel à la 
société FORMATO en tant que support technique. 
Les consultants et leurs compétences privilégiées sont :
\begin{itemize}
  \item Jérôme DELATOUR (spécification / conception) : jerome.delatour@eseo.fr
  \item Matthias BRUN (codage / tests) : matthias.brun@eseo.fr
  \item Camille CONSTANT (codage/tests) : camille.constant@eseo.fr
\end{itemize}
Des activités d’audits externes (cf. chapitre 5.3, page {\completer}) 
seront menées par les auditeurs de la société FORMATO ou missionnées par elle.

\subsection{Tâches du projet}
\subsubsection{Tâches transversales}
Les tâches transversales de l’assurance qualité incluent les activités suivantes :
\begin{itemize}
  \item Documentation (cf. chapitre 3, page 22).
  \item Revues et audits (cf. chapitre 5 , page 31 )
  \item Inspections internes
  \item Validation et tests
  \item Activités d’amélioration du processus d’AQ
\end{itemize}

\subsubsubsection{Inspections internes}
\completer

\subsubsubsection{Validation et test}
Le Plan de test a pour objectif d’identifier les informations existantes du projet 
et les composants qui doivent être testés. Il énumère les exigences d’évaluation à 
différents niveaux, décrit les stratégies de test qui seront employées, identifie 
les ressources nécessaires et met en évidence les biens livrables pour les tests.\\
\completer
\begin{itemize}
  \item Portée du document, termes et abréviations
  \item Références
  \item Périmètre de test (composants concernés ou non par les tests, 
  fonctionnalités testées ou non, critères d'acceptation des tests)
  \item Processus et stratégie de test(activités, techniques, outils, 
  procédures de test et gestion des anomalies)
  \item Infrastructure de test
  \item Documents de test et livrables
  \item Responsabilités
  \item Équipe de test
  \item Planning prévisionnel
\end{itemize}
Cf : [Plan de test \completer]

\subsubsubsection{Évolution et amélioration du PAQL}
Le PAQL est susceptible d'évoluer au cours du projet, en particulier pour les raisons suivantes :
\begin{itemize}
  \item Toutes les informations nécessaires à la rédaction d'un chapitre 
  ou d'un paragraphe ne sont pas connues ou suffisamment stabilisées lors de la rédaction.
  \item Il s'agit d'une phase du cycle de développement qui sera engagée ultérieurement 
  (cas de la mention « Rédaction réservée »).
  \item Des événements techniques ou organisationnels nécessitant une prise en 
  compte dans le PAQL peuvent apparaître lors du déroulement du projet (modification 
  d'organisation, mise en place de nouvelles normes ou de procédures ou modification 
  de normes ou procédures existantes, ...).
\end{itemize}
Le PAQL est rédigé par le Responsable Qualité et Test (RQT) de l’équipe projet. 
Le Chef de Projet (CdP) et les RQT participent aux décisions de modifications. 
Il incombe au RQT d’effectuer les modifications jugées nécessaires du PAQL. 
En cas de modifications du PAQL, celui-ci devra être signé à nouveau par les 
membres de l’équipe projet.

\subsubsection{Tâches par rapport au cycle de développement}
L’équipe projet suivra un cycle de développement en V en deux incréments. 
Les activités d’AQ sont décrites par rapport à ce cycle. 
Le planning et les échéances associées sont disponibles sur 
l’Espace Numérique de Travail du Projet (ENTP).
\begin{figure}[H]
  \centering
  \includegraphics[width=15cm]{cycle.png}
  \caption{Diagramme du cycle de développement du projet}
\end{figure}

\subsubsubsection{Phase d'initialisation du projet}
\begin{table}[H]
  \renewcommand{\arraystretch}{1.1}
  \begin{tabular}{|lll|}
  \hline
  \rowcolor[HTML]{CCCCCC} 
  \multicolumn{3}{|l|}{\cellcolor[HTML]{CCCCCC}\textbf{Phase : Initialisation}}                                                                                                                                                                                                                                                                   \\ \hline
  \multicolumn{3}{|l|}{
    \begin{tabular}[c]{@{}l@{}}
      Objectifs :\\ 
      Prendre en charge le projet, l’organiser, le planifier et en valider les bases.\\ 
      Évaluer les actions nécessaires pour mettre en place le projet.\\ 
      Échanger avec l'équipe sur les règles à définir.
    \end{tabular}}                                                                                                                                                                                                                                 \\ \hline
  \multicolumn{3}{|l|}{
    \begin{tabular}[c]{@{}l@{}}
      Remarques :\\
    \end{tabular}}                                                                                                                                                                                                                                                                \\ \hline
  \multicolumn{1}{|l|}{
    \begin{tabular}[c]{@{}l@{}}
      Acteurs \& reponsabilités :\\ 
      \tabitem CdP et RQT
    \end{tabular}}&\multicolumn{1}{l|}{
    \begin{tabular}[c]{@{}l@{}}
      Méthodes \& Règles :\\ 
      \tabitem Règles pour l'utilisation\\ de l'ENTP\\
      \tabitem Anticipation et organisation\\des deadlines personnelles
    \end{tabular}}& 
    \begin{tabular}[c]{@{}l@{}}
      Moyens \& Outils :\\ 
      \tabitem Initialisation du projet\\ sous ENTP
    \end{tabular}\\ \hline
  \multicolumn{1}{|l|}{
    \begin{tabular}[c]{@{}l@{}}
      Activités \\ d'organisation/pilotage :\\ 
      \tabitem Organisation de la \\réunion de lancement\\ 
      \tabitem Organisation de la phase\\ en aval
    \end{tabular}}& \multicolumn{1}{l|}{
      \begin{tabular}[c]{@{}l@{}}
        Activités de \\ production/soutien :\\ 
        \tabitem Élaboration PAQL\\ 
        \tabitem Mise en place de l’ENTP\\
        \tabitem Définition de la démarche\\ du projet\\
        \tabitem Initialisation du planning \\et du suivi du projet
      \end{tabular}}& 
      \begin{tabular}[c]{@{}l@{}}
        Activités de \\ vérification/contrôle :\\ 
        \tabitem Réunion de lancement\\
        \tabitem \completer
      \end{tabular}\\ \hline
  \multicolumn{1}{|l|}{
    \begin{tabular}[c]{@{}l@{}}
      Produits/données en entrée :\\ 
      \tabitem Wiki Prose et RedMine\\ 
      \tabitem Documents pédagogiques\\
    \end{tabular}}& \multicolumn{1}{l|}{
      \begin{tabular}[c]{@{}l@{}}
        Produits/données en sortie :\\ 
        \tabitem Planning des tâches \\sur l'ENTP\\ 
      \end{tabular}}           & 
      \begin{tabular}[c]{@{}l@{}}
        Produits révisés :\\ 
        \tabitem PAQL
      \end{tabular}\\ \hline
  \multicolumn{3}{|l|}{\begin{tabular}[c]{@{}l@{}}
    Jalons de la phase :\\ 
    \tabitem J1 : \completer
  \end{tabular}}                                                                                                                                                                                                                                                    \\ \hline
  \rowcolor[HTML]{CCCCCC} 
  \multicolumn{1}{|l|}{\cellcolor[HTML]{CCCCCC}
  \begin{tabular}[c]{@{}l@{}}
    Conditions \\ de début de phase :
  \end{tabular}} & \multicolumn{1}{l|}{\cellcolor[HTML]{CCCCCC}
  \begin{tabular}[c]{@{}l@{}}
    Condition \\ de fin de phase :
  \end{tabular}} & 
  \begin{tabular}[c]{@{}l@{}}
    Conditions \\ de passage à la \\phase suivante :
  \end{tabular}\\ \hline
  \multicolumn{1}{|l|}{
    \begin{tabular}[c]{@{}l@{}}
      \tabitem Nomination des CdP et \\RQT
    \end{tabular}}& \multicolumn{1}{l|}{
      \begin{tabular}[c]{@{}l@{}}
        \tabitem Validation des futurs \\livrables\\
        \tabitem Validation de l'équipe \\du planning prévisionnel
      \end{tabular}}& 
      \begin{tabular}[c]{@{}l@{}}
        \tabitem ENTP opérationnel\\
        \tabitem Attribution des rôles
      \end{tabular}\\ \hline
  \end{tabular}
\end{table}

\subsubsubsection{Phase de spécification}
Spécifications : Le dossier de spécification devra respecter le plan défini par la norme 
\cite[IEEE\-830\_1998]{830} et s’appuyer sur la notation UML \cite[UML\_2.4\_2011]{UML}. 
Deux audits (un consultatif et un normatif) porteront sur le dossier de 
spécification. Le plan de test ainsi que le cahier de test de validation 
seront établis durant cette étape de spécification. Deux audits 
(un consultatif et un normatif) porteront sur cette activité. Une revue de 
mi-avancement aura lieu pour présenter au client le dossier de spécification 
et les éléments contractuels.

\begin{table}[H]
  \renewcommand{\arraystretch}{1.1}
  \begin{tabular}{|lll|}
  \hline
  \rowcolor[HTML]{CCCCCC} 
  \multicolumn{3}{|l|}{\cellcolor[HTML]{CCCCCC}\textbf{PHASE : SPECIFICATION V1 \& V2}}                                                                                                                                                                                                                                                                   \\ \hline
  \multicolumn{3}{|l|}{
    \begin{tabular}[c]{@{}l@{}}
      Objectifs :\\ 
      Mener des activités d’exploration techniques afin d’évaluer la complexité et le temps \\
      nécessaire à la réalisation du futur produit.\\ 
      Présenter les principales fonctions, les performances requises, les exigences de \\
      qualité et les contraintes de réalisation.\\ 
      Faire la description complète de toutes les fonctionnalités des sous-ensembles du projet.\\
	  Présenter le dossier final de spécification.\\
	  Présenter le plan de test.\\
	  Lancer les explorations techniques nécessaires au projet.
    \end{tabular}}                                                                                                                                                                                                                                 \\ \hline
  \multicolumn{3}{|l|}{
    \begin{tabular}[c]{@{}l@{}}
      Remarques :\\
    \end{tabular}}                                                                                                                                                                                                                                                                \\ \hline
  \multicolumn{1}{|l|}{
    \begin{tabular}[c]{@{}l@{}}
      Acteurs \& reponsabilités :\\ 
      \tabitem CdP et équipe\\
	  \tabitem Client
    \end{tabular}}&\multicolumn{1}{l|}{
    \begin{tabular}[c]{@{}l@{}}
      Méthodes \& Règles :\\ 
      \tabitem PAQL
    \end{tabular}}& 
    \begin{tabular}[c]{@{}l@{}}
      Moyens \& Outils :\\ 
      \tabitem ENTP
    \end{tabular}\\ \hline
  \multicolumn{1}{|l|}{
    \begin{tabular}[c]{@{}l@{}}
      Activités \\ d'organisation/pilotage :\\ 
      \tabitem Organiser les échanges \\d’informations avec le client 
    \end{tabular}}& \multicolumn{1}{l|}{
      \begin{tabular}[c]{@{}l@{}}
        Activités de \\ production/soutien :\\ 
        \tabitem Rédaction du dossier\\ de spécifications\\ 
        \tabitem Rédaction du plan\\ de test\\
        \tabitem Élaboration des \\maquettes des écrans\\
        \tabitem Explorations technique\\
		\tabitem \completer
      \end{tabular}}& 
      \begin{tabular}[c]{@{}l@{}}
        Activités de \\ vérification/contrôle :\\ 
        \tabitem AC sur spécification\\
        \tabitem \completer
      \end{tabular}\\ \hline
  \multicolumn{1}{|l|}{
    \begin{tabular}[c]{@{}l@{}}
      Produits/données en entrée :\\ 
      \tabitem Cahier des charges Client\\ 
      \tabitem PAQL\\
    \end{tabular}}& \multicolumn{1}{l|}{
      \begin{tabular}[c]{@{}l@{}}
        Produits/données en sortie :\\ 
        \tabitem Dossier de spécifications\\
		\tabitem Plan de test\\
		\tabitem Contrat client\\
      \end{tabular}}           & 
      \begin{tabular}[c]{@{}l@{}}
        Produits révisés :\\ 
        \tabitem ENTP\\
		    \tabitem Planning prévisionnel\\
		    \tabitem Plan de test\\
		    \tabitem PAQL
      \end{tabular}\\ \hline
  \multicolumn{3}{|l|}{\begin{tabular}[c]{@{}l@{}}
    Jalons de la phase :\\ 
    \tabitem J1 : \completer \\
	\tabitem \\
	\tabitem \\
  \end{tabular}}                                                                                                                                                                                                                                                    \\ \hline
  \rowcolor[HTML]{CCCCCC} 
  \multicolumn{1}{|l|}{\cellcolor[HTML]{CCCCCC}
  \begin{tabular}[c]{@{}l@{}}
    Conditions \\ de début de phase :
  \end{tabular}} & \multicolumn{1}{l|}{\cellcolor[HTML]{CCCCCC}
  \begin{tabular}[c]{@{}l@{}}
    Condition \\ de fin de phase :
  \end{tabular}} & 
  \begin{tabular}[c]{@{}l@{}}
    Conditions \\ de passage à la \\phase suivante :
  \end{tabular}\\ \hline
  \multicolumn{1}{|l|}{
    \begin{tabular}[c]{@{}l@{}}
      \tabitem Initialisation effectuée
    \end{tabular}}& \multicolumn{1}{l|}{
    \begin{tabular}[c]{@{}l@{}}
      \tabitem Validation des produits\\ en sortie
      \tabitem Signature du client\\
    \end{tabular}}& 
    \begin{tabular}[c]{@{}l@{}}
      \tabitem Dossier de spécifications \\signé par le client\\
      \tabitem Plan de test validé\\
    \end{tabular}\\ \hline
  \end{tabular}
\end{table}

\subsubsubsection{Phase de conception}
Conception générale (ou système) : Un audit consultatif 
portera sur la conception générale.\\
Conception détaillée : Un audit consultatif portera sur la conception détaillée.\\
Un audit normatif portera sur la conception générale et détaillée.

\begin{table}[H]
  \renewcommand{\arraystretch}{1.1}
  \begin{tabular}{|lll|}
  \hline
  \rowcolor[HTML]{CCCCCC} 
  \multicolumn{3}{|l|}{\cellcolor[HTML]{CCCCCC}\textbf{PHASE : Conception V1 \& V2}}                                                                                                                                                                                                                                                                   \\ \hline
  \multicolumn{3}{|l|}{
    \begin{tabular}[c]{@{}l@{}}
      Objectifs :\\ 
      Organiser et optimiser les temps de conception.\\ 
      Définir l’architecture logicielle des applications.\\ 
      Finaliser les ultimes explorations techniques et les intégrer au dossier.\\
	  Décrire explicitement le dossier de conception.\\
    \end{tabular}}                                                                                                                                                                                                                                 \\ \hline
  \multicolumn{3}{|l|}{
    \begin{tabular}[c]{@{}l@{}}
      Remarques :\\
	  \tabitem La conception générale peut commencer en parallèle des spécifications.
    \end{tabular}}                                                                                                                                                                                                                                                                \\ \hline
  \multicolumn{1}{|l|}{
    \begin{tabular}[c]{@{}l@{}}
      Acteurs \& reponsabilités :\\ 
      \tabitem CdP et équipe\\
	  \tabitem Client
    \end{tabular}}&\multicolumn{1}{l|}{
    \begin{tabular}[c]{@{}l@{}}
      Méthodes \& Règles :\\ 
      \tabitem PAQL\\
	  \tabitem Dossier de spécification\\
    \end{tabular}}& 
    \begin{tabular}[c]{@{}l@{}}
      Moyens \& Outils :\\ 
      \tabitem ENTP
    \end{tabular}\\ \hline
  \multicolumn{1}{|l|}{
    \begin{tabular}[c]{@{}l@{}}
      Activités \\ d'organisation/pilotage :\\ 
      \tabitem Organiser les échanges \\d’informations avec le client 
    \end{tabular}}& \multicolumn{1}{l|}{
      \begin{tabular}[c]{@{}l@{}}
        Activités de \\ production/soutien :\\ 
        \tabitem Définition de l’architecture\\ technique\\
        \tabitem Conception support de \\communication \\Rasberry/IPX800V4\\
        \tabitem Conception logicielle\\
        \tabitem Normalisation des fonctions\\
		\tabitem Rédaction du dossier de \\conception\\
		\tabitem Rédaction des tests de \\validation\\
		\tabitem \completer
      \end{tabular}}& 
      \begin{tabular}[c]{@{}l@{}}
        Activités de \\ vérification/contrôle :\\ 
        \tabitem \completer
      \end{tabular}\\ \hline
  \multicolumn{1}{|l|}{
    \begin{tabular}[c]{@{}l@{}}
      Produits/données en entrée :\\ 
      \tabitem Dossier de spécifications\\ 
      \tabitem PAQL\\
    \end{tabular}}& \multicolumn{1}{l|}{
      \begin{tabular}[c]{@{}l@{}}
        Produits/données en sortie :\\ 
        \tabitem Dossier de conception\\
      \end{tabular}}           & 
      \begin{tabular}[c]{@{}l@{}}
        Produits révisés :\\ 
        \tabitem Normes de développement\\
		\tabitem Dossier de spécifications\\
		\tabitem PAQL
      \end{tabular}\\ \hline
  \multicolumn{3}{|l|}{\begin{tabular}[c]{@{}l@{}}
    Jalons de la phase :\\ 
    \tabitem J1 : \completer \\
  \end{tabular}}                                                                                                                                                                                                                                                    \\ \hline
  \rowcolor[HTML]{CCCCCC} 
  \multicolumn{1}{|l|}{\cellcolor[HTML]{CCCCCC}
  \begin{tabular}[c]{@{}l@{}}
    Conditions \\ de début de phase :
  \end{tabular}} & \multicolumn{1}{l|}{\cellcolor[HTML]{CCCCCC}
  \begin{tabular}[c]{@{}l@{}}
    Condition \\ de fin de phase :
  \end{tabular}} & 
  \begin{tabular}[c]{@{}l@{}}
    Conditions \\ de passage à la \\phase suivante :
  \end{tabular}\\ \hline
  \multicolumn{1}{|l|}{
    \begin{tabular}[c]{@{}l@{}}
      \tabitem \completer
    \end{tabular}}& \multicolumn{1}{l|}{
      \begin{tabular}[c]{@{}l@{}}
        \tabitem Validation des produits\\ en sortie
      \end{tabular}}& 
      \begin{tabular}[c]{@{}l@{}}
        \tabitem Dossier de conception validé\\
        \tabitem Normes de développement\\
		\tabitem Tests de validation\\ rédigés
      \end{tabular}\\ \hline
  \end{tabular}
\end{table}

\subsubsubsection{Phase de réalisation}
Codage : Quatre audits porteront sur le code source produit afin notamment 
de s’assurer du bon respect du PAQL et des normes de programmation. 
Deux porteront sur le code écrit en langage C sur cible embarquée 
(consultatif et normatif) et deux autres sur le code fonctionnant sur la plate-forme Android.

\begin{table}[H]
  \renewcommand{\arraystretch}{1.1}
  \begin{tabular}{|lll|}
  \hline
  \rowcolor[HTML]{CCCCCC} 
  \multicolumn{3}{|l|}{\cellcolor[HTML]{CCCCCC}\textbf{PHASE : Réalisation V1 \& V2}}                                                                                                                                                                                                                                                                   \\ \hline
  \multicolumn{3}{|l|}{
    \begin{tabular}[c]{@{}l@{}}
      Objectifs :\\ 
      Développer des applications logicielles.\\ 
      Tester les applications logicielles définies lors de la phase de conception.\\ 
      Coder les tests d'intégration et unitaires et adopter les outils de test.\\
    \end{tabular}}                                                                                                                                                                                                                                 \\ \hline
  \multicolumn{3}{|l|}{
    \begin{tabular}[c]{@{}l@{}}
      Remarques :\\
	  \tabitem L’étape de réalisation peut commencer en parallèle de la conception.
    \end{tabular}}                                                                                                                                                                                                                                                                \\ \hline
  \multicolumn{1}{|l|}{
    \begin{tabular}[c]{@{}l@{}}
      Acteurs \& reponsabilités :\\ 
      \tabitem CdP et équipe\\
    \end{tabular}}&\multicolumn{1}{l|}{
    \begin{tabular}[c]{@{}l@{}}
      Méthodes \& Règles :\\ 
      \tabitem PAQL\\
	  \tabitem Dossier de spécification\\
	  \tabitem Dossier de conception\\
    \end{tabular}}& 
    \begin{tabular}[c]{@{}l@{}}
      Moyens \& Outils :\\ 
      \tabitem ENTP\\
	    \tabitem Moyen de test\\
      \tabitem Testlink \& Eclipse\\ 
    \end{tabular}\\ \hline
  \multicolumn{1}{|l|}{
    \begin{tabular}[c]{@{}l@{}}
      Activités \\ d'organisation/pilotage :\\ 
      \tabitem Organiser les fonctions \\prioritaires\\
	  \tabitem Organiser les échanges \\d'informations avec le client
    \end{tabular}}& \multicolumn{1}{l|}{
      \begin{tabular}[c]{@{}l@{}}
        Activités de \\ production/soutien :\\ 
		\tabitem \completer
      \end{tabular}}& 
      \begin{tabular}[c]{@{}l@{}}
        Activités de \\ vérification/contrôle :\\ 
        \tabitem \completer
      \end{tabular}\\ \hline
  \multicolumn{1}{|l|}{
    \begin{tabular}[c]{@{}l@{}}
      Produits/données en entrée :\\ 
      \tabitem Dossier de conception\\ 
      \tabitem PAQL\\
    \end{tabular}}& \multicolumn{1}{l|}{
      \begin{tabular}[c]{@{}l@{}}
        Produits/données en sortie :\\ 
        \tabitem Code\\
		\tabitem Moyen de test opérationnel\\
		\tabitem tests unitaires\\
		\tabitem \completer
      \end{tabular}}           & 
      \begin{tabular}[c]{@{}l@{}}
        Produits révisés :\\ 
        \tabitem Normes de développement\\
		\tabitem Dossier de spécifications\\
		\tabitem Dossier de conception\\
		\tabitem PAQL
      \end{tabular}\\ \hline
  \multicolumn{3}{|l|}{\begin{tabular}[c]{@{}l@{}}
    Jalons de la phase :\\ 
    \tabitem J1 : \completer \\
  \end{tabular}}                                                                                                                                                                                                                                                    \\ \hline
  \rowcolor[HTML]{CCCCCC} 
  \multicolumn{1}{|l|}{\cellcolor[HTML]{CCCCCC}
  \begin{tabular}[c]{@{}l@{}}
    Conditions \\ de début de phase :
  \end{tabular}} & \multicolumn{1}{l|}{\cellcolor[HTML]{CCCCCC}
  \begin{tabular}[c]{@{}l@{}}
    Condition \\ de fin de phase :
  \end{tabular}} & 
  \begin{tabular}[c]{@{}l@{}}
    Conditions \\ de passage à la \\phase suivante :
  \end{tabular}\\ \hline
  \multicolumn{1}{|l|}{
    \begin{tabular}[c]{@{}l@{}}
      \tabitem \completer
    \end{tabular}}& \multicolumn{1}{l|}{
      \begin{tabular}[c]{@{}l@{}}
        \tabitem Validation des produits\\ en sortie
      \end{tabular}}& 
      \begin{tabular}[c]{@{}l@{}}
        \tabitem Application logicielle \\réalisée\\
		\tabitem Tests réalisables
      \end{tabular}\\ \hline
  \end{tabular}
\end{table}

\subsubsubsection{Phase de test}
Tests unitaires, tests d’intégration : Un audit consultatif 
portera sur chacune de ces activités.\\
Tests de validation : Un audit normatif portera sur 
l’application des tests (validation, intégration et unitaire).\\

\begin{table}[H]
  \renewcommand{\arraystretch}{1.1}
  \begin{tabular}{|lll|}
  \hline
  \rowcolor[HTML]{CCCCCC} 
  \multicolumn{3}{|l|}{\cellcolor[HTML]{CCCCCC}\textbf{PHASE : Test V1 \& V2}}                                                                                                                                                                                                                                                                   \\ \hline
  \multicolumn{3}{|l|}{
    \begin{tabular}[c]{@{}l@{}}
      Objectifs :\\ 
      Développer des tests fiables et efficaces.\\ 
      Contrôler la fiabilité du logiciel.\\ 
      Identifier les erreurs logiques.\\
	  Vérifier interactions des interfaces.\\
	  Valider l'adéquation aux spécifications du logiciel.\\
    \end{tabular}}                                                                                                                                                                                                                                 \\ \hline
  \multicolumn{3}{|l|}{
    \begin{tabular}[c]{@{}l@{}}
      Remarques :\\
	  \tabitem La phase de programmation des tests peut commencer en parallèle de la réalisation
    \end{tabular}}                                                                                                                                                                                                                                                                \\ \hline
  \multicolumn{1}{|l|}{
    \begin{tabular}[c]{@{}l@{}}
      Acteurs \& reponsabilités :\\ 
      \tabitem CdP et Resp Test\\
	  \tabitem Équipe de test
    \end{tabular}}&\multicolumn{1}{l|}{
    \begin{tabular}[c]{@{}l@{}}
      Méthodes \& Règles :\\ 
      \tabitem PAQL\\
	  \tabitem Conception\\
	  \tabitem Spécification\\
	  \tabitem Plan de test\\
	  \tabitem Cahier de tests
    \end{tabular}}& 
    \begin{tabular}[c]{@{}l@{}}
      Moyens \& Outils :\\ 
      \tabitem ENTP\\
	    \tabitem Moyen de test\\
      \tabitem Testlink
    \end{tabular}\\ \hline
  \multicolumn{1}{|l|}{
    \begin{tabular}[c]{@{}l@{}}
      Activités \\ d'organisation/pilotage :\\ 
      \tabitem Vérification croisée\\
	  \tabitem Organiser les tests \\principaux\\
	  \tabitem \completer
    \end{tabular}}& \multicolumn{1}{l|}{
      \begin{tabular}[c]{@{}l@{}}
        Activités de \\ production/soutien :\\ 
		\tabitem Exécution tests de \\communication\\
		\tabitem Compléter cahier de test\\
		\tabitem Exécution  tests robustesse\\
		\tabitem Exécution  tests unitaires\\
		\tabitem Exécution  tests validation\\
      \end{tabular}}& 
      \begin{tabular}[c]{@{}l@{}}
        Activités de \\ vérification/contrôle :\\ 
        \tabitem \completer
      \end{tabular}\\ \hline
  \multicolumn{1}{|l|}{
    \begin{tabular}[c]{@{}l@{}}
      Produits/données en entrée :\\ 
      \tabitem Application logicielle\\ 
      \tabitem Dossier de conception\\
	  \tabitem PAQL\\
	  \tabitem Moyen de test\\
	  \tabitem Cahier de test\\
    \end{tabular}}& \multicolumn{1}{l|}{
      \begin{tabular}[c]{@{}l@{}}
        Produits/données en sortie :\\ 
        \tabitem Document d'analyse des tests\\
		\tabitem Application logicielle testée\\
		\tabitem Cahier de test\\
		\tabitem \completer
      \end{tabular}}           & 
      \begin{tabular}[c]{@{}l@{}}
        Produits révisés :\\ 
        \tabitem Normes de développement\\
		\tabitem Document de conception\\
		\tabitem Document de spécifications\\
		\tabitem Code
      \end{tabular}\\ \hline
  \multicolumn{3}{|l|}{\begin{tabular}[c]{@{}l@{}}
    Jalons de la phase :\\ 
    \tabitem J1 : \completer \\
  \end{tabular}}                                                                                                                                                                                                                                                    \\ \hline
  \rowcolor[HTML]{CCCCCC} 
  \multicolumn{1}{|l|}{\cellcolor[HTML]{CCCCCC}
  \begin{tabular}[c]{@{}l@{}}
    Conditions \\ de début de phase :
  \end{tabular}} & \multicolumn{1}{l|}{\cellcolor[HTML]{CCCCCC}
  \begin{tabular}[c]{@{}l@{}}
    Condition \\ de fin de phase :
  \end{tabular}} & 
  \begin{tabular}[c]{@{}l@{}}
    Conditions \\ de passage à la \\phase suivante :
  \end{tabular}\\ \hline
  \multicolumn{1}{|l|}{
    \begin{tabular}[c]{@{}l@{}}
      \tabitem Plan de test \\partiellement validé
    \end{tabular}}& \multicolumn{1}{l|}{
    \begin{tabular}[c]{@{}l@{}}
      \tabitem Validation des produits\\ en sortie
    \end{tabular}}& 
    \begin{tabular}[c]{@{}l@{}}
      \tabitem Tests réalisés\\
      \tabitem Cahier de test complet
    \end{tabular}\\ \hline
  \end{tabular}
\end{table}

\subsubsubsection{Phase de recette}
Une revue de recette aura lieu pour remettre au client le produit demandé, 
ainsi que fournir les livrables suivants : les dossiers de spécification 
et conception, l'ensemble des tests réalisés (cahier et plan de test), 
les codes sources, les manuels d'utilisation et d'installation, le procès 
verbal de recette définitive ainsi que l'application logicielle fonctionnelle.

\begin{table}[H]
  \renewcommand{\arraystretch}{1.1}
  \begin{tabular}{|lll|}
  \hline
  \rowcolor[HTML]{CCCCCC} 
  \multicolumn{3}{|l|}{\cellcolor[HTML]{CCCCCC}\textbf{PHASE : Recette}}                                                                                                                                                                                                                                                                   \\ \hline
  \multicolumn{3}{|l|}{
    \begin{tabular}[c]{@{}l@{}}
      Objectifs :\\ 
      Réceptionner les applications de manière définitive\\
    \end{tabular}}                                                                                                                                                                                                                                 \\ \hline
  \multicolumn{3}{|l|}{
    \begin{tabular}[c]{@{}l@{}}
      Remarques :\\
	  \tabitem Se fera en même temps que la phase de bilan de fin de projet
    \end{tabular}}                                                                                                                                                                                                                                                                \\ \hline
  \multicolumn{1}{|l|}{
    \begin{tabular}[c]{@{}l@{}}
      Acteurs \& reponsabilités :\\ 
      \tabitem CdP et équipe\\
	  \tabitem Client
    \end{tabular}}&\multicolumn{1}{l|}{
    \begin{tabular}[c]{@{}l@{}}
      Méthodes \& Règles :\\ 
      \tabitem PAQL\\
	  \tabitem Conception\\
	  \tabitem Spécification\\
	  \tabitem Intégration
    \end{tabular}}& 
    \begin{tabular}[c]{@{}l@{}}
      Moyens \& Outils :\\ 
      \tabitem ENTP
    \end{tabular}\\ \hline
  \multicolumn{1}{|l|}{
    \begin{tabular}[c]{@{}l@{}}
      Activités \\ d'organisation/pilotage :\\ 
      \tabitem Organiser les échanges \\d’informations avec le client
    \end{tabular}}& \multicolumn{1}{l|}{
      \begin{tabular}[c]{@{}l@{}}
        Activités de \\ production/soutien :\\ 
		\tabitem \completer		
      \end{tabular}}& 
      \begin{tabular}[c]{@{}l@{}}
        Activités de \\ vérification/contrôle :\\ 
        \tabitem \completer
      \end{tabular}\\ \hline
  \multicolumn{1}{|l|}{
    \begin{tabular}[c]{@{}l@{}}
      Produits/données en entrée :\\ 
      \tabitem PAQL\\
	  \tabitem Dossier de specification\\
	  \tabitem Dossier de conception\\
	  \tabitem Cahier de test\\
	  \tabitem Code
    \end{tabular}}& \multicolumn{1}{l|}{
      \begin{tabular}[c]{@{}l@{}}
        Produits/données en sortie :\\ 
        \tabitem Application opérationnelle \\livrée\\
		\tabitem Manuel d’utilisation et \\manuel d’installation validés\\
		\tabitem Dernière version à jour \\des documents, codes \\source des applications \\(documenté sous Doxygen) \\et exécutables\\
		\tabitem Procès verbal de recette \\définitive
      \end{tabular}}           & 
      \begin{tabular}[c]{@{}l@{}}
        Produits révisés :\\ 
        \tabitem Cahier de test\\
		\tabitem Manuel d’utilisation\\
		\tabitem Manuel d’installation\\
		\tabitem Code source\\
		\tabitem Recette\\
		\tabitem \completer
      \end{tabular}\\ \hline
  \multicolumn{3}{|l|}{\begin{tabular}[c]{@{}l@{}}
    Jalons de la phase :\\ 
    \tabitem J1 : \completer \\
  \end{tabular}}                                                                                                                                                                                                                                                    \\ \hline
  \rowcolor[HTML]{CCCCCC} 
  \multicolumn{1}{|l|}{\cellcolor[HTML]{CCCCCC}
  \begin{tabular}[c]{@{}l@{}}
    Conditions \\ de début de phase :
  \end{tabular}} & \multicolumn{1}{l|}{\cellcolor[HTML]{CCCCCC}
  \begin{tabular}[c]{@{}l@{}}
    Condition \\ de fin de phase :
  \end{tabular}} & 
  \begin{tabular}[c]{@{}l@{}}
    Conditions \\ de passage à la \\phase suivante :
  \end{tabular}\\ \hline
  \multicolumn{1}{|l|}{
    \begin{tabular}[c]{@{}l@{}}
      \tabitem Phase d’intégration terminée
    \end{tabular}}& \multicolumn{1}{l|}{
    \begin{tabular}[c]{@{}l@{}}
      \tabitem Validation des produits\\ en sortie
    \end{tabular}}& 
    \begin{tabular}[c]{@{}l@{}}
      \tabitem Fin du projet\\
    \end{tabular}\\ \hline
  \end{tabular}
\end{table}

\subsubsubsection{Phase de bilan de fin de projet}

\begin{table}[H]
  \renewcommand{\arraystretch}{1.1}
  \begin{tabular}{|lll|}
  \hline
  \rowcolor[HTML]{CCCCCC} 
  \multicolumn{3}{|l|}{\cellcolor[HTML]{CCCCCC}\textbf{PHASE : Bilan Fin de Projet}}\\ \hline
  \multicolumn{3}{|l|}{
    \begin{tabular}[c]{@{}l@{}}
      Objectifs :\\ 
      Clôturer le projet, archiver les productions, effectuez les statistiques du projet (temps passé), \\
      tirer les enseignements du projet.\\
    \end{tabular}}\\ \hline
  \multicolumn{3}{|l|}{
    \begin{tabular}[c]{@{}l@{}}
      Remarques :\\
	  \tabitem Se fera en même temps que la phase de recette.
    \end{tabular}}\\ \hline
  \multicolumn{1}{|l|}{
    \begin{tabular}[c]{@{}l@{}}
      Acteurs \& reponsabilités :\\ 
      \tabitem CdP\\
    \end{tabular}}&\multicolumn{1}{l|}{
    \begin{tabular}[c]{@{}l@{}}
      Méthodes \& Règles :\\ 
      \tabitem PAQL\\
    \end{tabular}}& 
    \begin{tabular}[c]{@{}l@{}}
      Moyens \& Outils :\\ 
      \tabitem ENTP\\
    \end{tabular}\\ \hline
  \multicolumn{1}{|l|}{
    \begin{tabular}[c]{@{}l@{}}
      Activités \\ d'organisation/pilotage :\\ 
      \tabitem Organiser les échanges \\d’informations avec le référent \\ESEO et M. Delatour
    \end{tabular}}& \multicolumn{1}{l|}{
      \begin{tabular}[c]{@{}l@{}}
        Activités de \\ production/soutien :\\ 
		\tabitem Rédaction des documents\\ de bilan\\	
      \end{tabular}}& 
      \begin{tabular}[c]{@{}l@{}}
        Activités de \\ vérification/contrôle :\\ 
        \tabitem Audit Interne\\
		\tabitem \completer
      \end{tabular}\\ \hline
  \multicolumn{1}{|l|}{
    \begin{tabular}[c]{@{}l@{}}
      Produits/données en entrée :\\ 
      \tabitem ENTP\\
	  \tabitem Livrables client\\
	  \tabitem RDP\\
    \end{tabular}}& \multicolumn{1}{l|}{
      \begin{tabular}[c]{@{}l@{}}
        Produits/données en sortie :\\ 
        \tabitem Statistiques projet\\
		    \tabitem Compte-rendu de réunion\\ de bilan de projet\\
		    \tabitem Synthèse enseignements projets\\
		    \tabitem Propositions amélioration\\
      \end{tabular}}           & 
      \begin{tabular}[c]{@{}l@{}}
        Produits révisés :\\ 
        \tabitem ENTP\\
		\tabitem Livrables\\
		\tabitem RDP\\
      \end{tabular}\\ \hline
  \multicolumn{3}{|l|}{\begin{tabular}[c]{@{}l@{}}
    Jalons de la phase :\\ 
    \tabitem J1 : \completer \\
  \end{tabular}}      \\ \hline
  \rowcolor[HTML]{CCCCCC} 
  \multicolumn{1}{|l|}{\cellcolor[HTML]{CCCCCC}
  \begin{tabular}[c]{@{}l@{}}
    Conditions \\ de début de phase :
  \end{tabular}} & \multicolumn{1}{l|}{\cellcolor[HTML]{CCCCCC}
  \begin{tabular}[c]{@{}l@{}}
    Condition \\ de fin de phase :
  \end{tabular}} & 
  \begin{tabular}[c]{@{}l@{}}
    Conditions \\ de passage à la \\phase suivante :
  \end{tabular}\\ \hline
  \multicolumn{1}{|l|}{
    \begin{tabular}[c]{@{}l@{}}
      \tabitem Phase d’intégration terminée
    \end{tabular}}& \multicolumn{1}{l|}{
    \begin{tabular}[c]{@{}l@{}}
      \tabitem Validation des produits\\ en sortie
    \end{tabular}}& 
    \begin{tabular}[c]{@{}l@{}}
      \tabitem Fin du projet\\
    \end{tabular}\\ \hline
  \end{tabular}
\end{table}

\subsection{Responsabilité}
\subsubsection{Définition Générale des rôles}

Chaque membre de l’équipe projet est tenu de respecter et d’appliquer les 
normes du PAQL dans son travail.\\
Chaque membre pourra avoir l’un des rôles suivants :
\begin{itemize}
  \item \textbf{Chef de projet (CdP) :} \completer
  \item \textbf{Responsable Qualité et Test (RQT) :} \completer
  \item \textbf{Développeur :} \completer
  \item \textbf{Responsable Développeur C | Java (RD) :} \completer
\end{itemize}

\subsubsection{Récapitulatif des responsabilités client sur les phases}

\begin{table}[H]
  \begin{tabular}{|l|l|l|}
  \hline
  \rowcolor[HTML]{CCCCCC} 
  \textbf{Phase} &
    \textbf{Implication Client} &
    \textbf{Implication Équipe Projet} \\ \hline
  \textbf{Initialisation} &
    \completer &
    \tabitem \completer \\ \hline
  \textbf{Spécification} &
    \begin{tabular}[c]{@{}l@{}}
      \tabitem Répondre aux questions \\concernant l'étude des besoins\\ 
      \tabitem Valider l'IHM\\ 
      \tabitem Fournir l'existant système \\si nécessaire\\ 
      \tabitem Valider les dossiers
    \end{tabular} 
      &
    \begin{tabular}[c]{@{}l@{}}
      \tabitem Réaliser l'étude des besoins\\ 
      \tabitem Rédiger le dossier de spécifications\\ 
      \tabitem Rédiger le cahier de recette
    \end{tabular} \\ \hline
  \textbf{Conception} &
    \tabitem \completer &
    \begin{tabular}[c]{@{}l@{}}
      \tabitem Rédiger le dossier de conception\\ 
      \tabitem Rédiger le cahier de test
    \end{tabular} \\ \hline
  \textbf{Réalisation} &
    \tabitem \completer &
    \begin{tabular}[c]{@{}l@{}}
      \tabitem Développer les modules logiciels\\ 
      \tabitem Tests unitaires de l'application
    \end{tabular} \\ \hline
  \textbf{Test} &
    \tabitem \completer &
    \begin{tabular}[c]{@{}l@{}}
      \tabitem Analyse des tests\\ 
      \tabitem Vérification finale de l'application
    \end{tabular} \\ \hline
  \textbf{Recette} &
    \begin{tabular}[c]{@{}l@{}}
      \tabitem Valider le taux de \\fiabilité des tests\\ 
      \tabitem Valider le fonctionnement \\global de l'application\\ 
      \tabitem Valider l'ensemble des \\livrables reçus\\ 
      \tabitem Signature du procès-verbal \\de recette définitive
    \end{tabular} &
    \begin{tabular}[c]{@{}l@{}}
      \tabitem Rédaction du procès-verbal de recette\\ 
      \tabitem Livrer l'ensemble des livrables
    \end{tabular} \\ \hline
  \textbf{\begin{tabular}[c]{@{}l@{}}Bilan de fin\\ de projet\end{tabular}} &
    \tabitem \completer &
    \begin{tabular}[c]{@{}l@{}}
      \tabitem Faire une analyse et \\les statistiques du projet\\
      \tabitem Proposer des améliorations\\ 
      \tabitem Rédiger un rapport de fin de projet
    \end{tabular} \\ \hline
  \end{tabular}
  \end{table}

\subsubsection{Récapitulatif des responsabilités CdP sur les phases}

\begin{table}[H]
  \begin{tabular}{ll}
  \rowcolor[HTML]{CCCCCC} 
  \textbf{Phase}          & \textbf{Implication CdP}                                                          \\
  \textbf{Initialisation} & \begin{tabular}[c]{@{}l@{}}\tabitem \completer\\ \tabitem \completer\end{tabular} \\
  \textbf{Spécification}  & \begin{tabular}[c]{@{}l@{}}\tabitem \completer\\ \tabitem \completer\end{tabular} \\
  \textbf{Conception}     & \begin{tabular}[c]{@{}l@{}}\tabitem \completer\\ \tabitem \completer\end{tabular} \\
  \textbf{Réalisation}    & \begin{tabular}[c]{@{}l@{}}\tabitem \completer\\ \tabitem \completer\end{tabular} \\
  \textbf{Test}           & \begin{tabular}[c]{@{}l@{}}\tabitem \completer\\ \tabitem \completer\end{tabular} \\
  \textbf{Recette}        & \begin{tabular}[c]{@{}l@{}}\tabitem \completer\\ \tabitem \completer\end{tabular} \\
  \textbf{\begin{tabular}[c]{@{}l@{}}Bilan de fin\\ de projet\end{tabular}} & \begin{tabular}[c]{@{}l@{}}\tabitem \completer\\ \tabitem \completer\end{tabular}
  \end{tabular}
\end{table}

\subsubsection{Récapitulatif des responsabilités RQ sur les phases}

\begin{table}[H]
  \begin{tabular}{|l|l|}
  \rowcolor[HTML]{CCCCCC} 
  \textbf{Phase}          & \textbf{Implication RQT}                                                          \\
  \textbf{Initialisation} & \begin{tabular}[c]{@{}l@{}}\tabitem \completer\\ \tabitem \completer\end{tabular} \\
  \textbf{Spécification}  & \begin{tabular}[c]{@{}l@{}}\tabitem \completer\\ \tabitem \completer\end{tabular} \\
  \textbf{Conception}     & \begin{tabular}[c]{@{}l@{}}\tabitem \completer\\ \tabitem \completer\end{tabular} \\
  \textbf{Réalisation}    & \begin{tabular}[c]{@{}l@{}}\tabitem \completer\\ \tabitem \completer\end{tabular} \\
  \textbf{Test}           & \begin{tabular}[c]{@{}l@{}}\tabitem \completer\\ \tabitem \completer\end{tabular} \\
  \textbf{Recette}        & \begin{tabular}[c]{@{}l@{}}\tabitem \completer\\ \tabitem \completer\end{tabular} \\
  \textbf{\begin{tabular}[c]{@{}l@{}}Bilan de fin\\ de projet\end{tabular}} & \begin{tabular}[c]{@{}l@{}}\tabitem \completer\\ \tabitem \completer\end{tabular}
  \end{tabular}
\end{table}

\subsubsection{Récapitulatif des responsabilités des développeurs sur les phases}

\begin{table}[H]
  \begin{tabular}{ll}
  \rowcolor[HTML]{CCCCCC} 
  \textbf{Phase}          & \textbf{Implication Équipe de Développement}                                                          \\
  \textbf{Initialisation} & \begin{tabular}[c]{@{}l@{}}\tabitem \completer\\ \tabitem \completer\end{tabular} \\
  \textbf{Spécification}  & \begin{tabular}[c]{@{}l@{}}\tabitem \completer\\ \tabitem \completer\end{tabular} \\
  \textbf{Conception}     & \begin{tabular}[c]{@{}l@{}}\tabitem \completer\\ \tabitem \completer\end{tabular} \\
  \textbf{Réalisation}    & \begin{tabular}[c]{@{}l@{}}\tabitem \completer\\ \tabitem \completer\end{tabular} \\
  \textbf{Test}           & \begin{tabular}[c]{@{}l@{}}\tabitem \completer\\ \tabitem \completer\end{tabular} \\
  \textbf{Recette}        & \begin{tabular}[c]{@{}l@{}}\tabitem \completer\\ \tabitem \completer\end{tabular} \\
  \textbf{\begin{tabular}[c]{@{}l@{}}Bilan de fin\\ de projet\end{tabular}} & \begin{tabular}[c]{@{}l@{}}\tabitem \completer\\ \tabitem \completer\end{tabular}
  \end{tabular}
\end{table}

\section{Documentation}
\subsection{But}
Ce chapitre décrit les règles de gestion de la documentation du projet. 
En effet, un certain nombre d’artefacts du projet concerne des documents.

\subsection{Type de documents}
Les documents suivants sont distingués suivant leur nature, 
qu’ils soient livrés par le client ou non, consultables par 
les auditeurs ou réservés à l’équipe projet, et ce suivant 
la phase où ils sont produits.\\
Les artefacts de documentation, nommés documents "livrables client" sont :
\begin{itemize}
  \item Dossier de spécification
  \item Plan d'Assurance Qualité Logicielle (PAQL)
  \item Contrat et devis
  \item Dossier de conception
  \item Plan de tests
  \item Artefacts de code
  \item Cahier de tests
  \item Manuel d'utilisation et manuel d'installation
  \item Présentation revue
\end{itemize}
Les artefacts consultables par les consultants, 
nommés documents "consultables auditeur", sont :
\begin{itemize}
  \item Tous les documents "livrables client"
  \item Ordre du jour réunion projet
  \item Compte-rendu réunion client
  \item Compte-rendu réunion projet
  \item Correspondance échangée avec le client
  \item Les codes sources des explorations techniques
  \item Les codes sources
  \item Les codes de test
\end{itemize}
Les autres documents du projet sont considérés comme internes au projet.

\subsection{Référence des documents}
La référence d’un document est de la forme suivante : 
« SIGLE\_EXX », où XX désigne l’identifiant de l’équipe 
(une lettre et un numéro) et le SIGLE correspond à 
l’une des combinaisons de lettres citées ci-dessous.\\

Ce système de référencement ne sera appliqué que pour 
les livrables "consultables auditeur". Les ébauches et 
documents internes à l'équipe échappent donc 
à cette règle de nommage.\\

\begin{table}[H]
  \begin{tabular}{|l|l|}
  \hline
  \rowcolor[HTML]{CCCCCC} 
  \textbf{Réference}           & \textbf{Libellé Document}           \\ \hline
  PAQL\_E{\teamNumber}          & Plan d'Assurance Qualité Logicielle \\ \hline
  SPEC\_E{\teamNumber}          & Document de spécification           \\ \hline
  CONC\_E{\teamNumber}          & Document de conception              \\ \hline
  TEST\_E{\teamNumber}          & Document de test                    \\ \hline
  BILAN\_E{\teamNumber}         & Bilan de projet                     \\ \hline
  \completer                   & \completer                          \\ \hline
  \completer                   & \completer                          \\ \hline
  \completer                   & \completer                          \\ \hline
  \completer                   & \completer                          \\ \hline
  \completer                   & \completer                          \\ \hline
  SANS\_E{\teamNumber}          & Pour tous les autres documents      \\ \hline
  \end{tabular}
\end{table}
\tabitem \completer\\
Par exemple, pour le dossier de spécification, produit par l’équipe XX, 
sa référence sera la suivante : « SPEC\_EXX ».\\
La référence d’un document permet d’identifier la catégorie à laquelle 
appartient ce document, cette référence figurera dans le 
document, mais aussi dans le nom de son fichier.\\
Ainsi, dans l’exemple précédent, le fichier 
pourra se nommer "dossier\_de\_specification\_SPEC\_EXX".


\subsection{État d’un document}
\subsection{Responsable du document}
\subsection{Processus d’édition d’un document}
\subsection{version d’un document}
\subsection{Format des documents}
\subsubsection{Modèle de document}
\subsubsection{Artefact de code}
\subsubsection{Règles de codage en langage C :}
\subsubsection{Règles de traduction de la conception vers du code en langage C}
\subsubsection{Règles de codage en langage Java}
\subsubsection{Règles de traduction de la conception vers du code en Java}
\subsection{documents internes}
\section{Standards, pratiques, conventions et métriques}
\subsection{But}
\subsection{Exigences qualités générales}
\subsection{Exigences qualités sur les artefacts}
\subsubsection{Exigences sur les documents consultables par les auditeurs}
\subsubsection{Exigences sur les documents livrables}
\subsubsection{Exigences sur le code source}
\section{Revues et Audits}
\subsection{But}
\subsection{Revues}
\subsubsection{Revue de mi-avancement}
\subsubsection{Revue de recette}
\subsection{Audits}
\subsubsection{Audit consultatif}
\subsubsection{Audit normatif}
\subsubsection{Inspection et revue croisée}
\section{Test}
\section{Notification des problèmes et corrections}
\section{Outils, Techniques et Methodologie}
\subsection{L’espace Numérique de Travail du Projet (ENTP)}
\subsubsection{Redmine}
\subsubsection{Planning prévisionnel}
\subsubsection{Suivi du travail}
\subsection{Liste des outils autorisés}
\section{Contrôle des médias}
\subsection{Communiquer entre membres internes du projet}
\subsection{Gestion des médias, sources, références, copyright}
\subsection{Communiquer avec des membres externes au projet}
\subsubsection{Contacter les consultants}
\subsubsection{Contacter le client}
\section{Contrôle des fournisseur}
\subsection{S'inscrire à une formation}
\subsection{Demander un consulting}
\section{Collecte, maintenance et conservation des archives}
\subsection{Le Référentiel Documentaire Projet (RDP)}
\subsubsection{Structuration du RDP}
\subsubsection{Obtenir un accès au RDP}
\subsubsection{Nom des fichiers des artefacts}
\subsubsection{Déposer un artefact sur le RDP}
\subsection{Gestion des documents papier}
\section{Formation}
\section{Gestion du risque}
\section{Outils et configurations}
\subsection{Réseau informatique}
\subsection{Subversion}
\section{Glossaire : Définitions, acronymes et abréviations}
\section{Validation du document}


\end{document}



%--- END 

% \end{document}