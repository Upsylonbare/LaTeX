%-----------------------------------
% Plan d'assurance qualité logicielle
% Originalement écrit par Jérôme Delatour
% Adapté en LaTeX par Clément Le Goffic en utilisant des artefacts de code de Camille Constant et Thomas Cravic
% ProSE
%
% Auteurs : J. Delatour, C. Constant, T.Cravic, C. Le Goffic
%
%-----------------------------------

\documentclass[a4paper,11pt,titlepage]{article}

\usepackage[english,french]{babel}
\usepackage[T1]{fontenc}
\usepackage[utf8]{inputenc}

\usepackage{xspace, graphicx}
\usepackage{amsmath}
\usepackage{amsfonts}
\usepackage{amssymb}
\usepackage{lastpage}
\usepackage{array}
\usepackage{tabularx}
\usepackage{hyperref}

\usepackage{titlesec}
\usepackage{float}
\usepackage[table,xcdraw]{xcolor}
\usepackage{color, soul}
\usepackage{fancyhdr}

%------------------------------------
%              Variables
%              You must complete all these parameters to personnalize your document
%------------------------------------
\newcommand{\version}{0.0}
\newcommand{\revision}{0}
\newcommand{\projectName}{Nom du Projet}
\newcommand{\documentName}{Nom du document}
\newcommand{\creator}{Nom du Créateur}
\newcommand{\creatorAbrev}{Abrev Créateur}
\newcommand{\documentNameAbrev}{XXXX}
\newcommand{\annee}{2022}
\newcommand{\teamName}{Nom d'équipe}
\newcommand{\prose}{ProSE}
\newcommand{\teamNumber}{A?}
%------------------------------------

%------------------------------------
%              Useful command 
%------------------------------------
\newcommand{\tabitem}{~~\llap{\textbullet}~~}
\newcommand{\completer}{\textbf{A compléter}} %Write "A completer when needed"
\newcommand{\ti}[1]{\begin{tabular}[c]{@{}l@{}}#1\end{tabular}} %Create a tab cell that takes cell text as parameter
%What follows only concern double columns table :
\newcommand{\tl}[2]{\ti{#1} & \ti{#2}} %Create a tab line that take two cell text as parameters
%------------------------------------

%------------------------------------
%              Parameters for header and footer
%------------------------------------
\pagestyle{fancy}
\graphicspath{{./figs/}}
\setlength{\hoffset}{-40pt}
\setlength{\topmargin}{-25pt}
\setlength{\headsep}{10pt}
\renewcommand{\headheight}{80pt}
\renewcommand{\headwidth}{450pt}
\setlength{\textwidth}{450pt}
\setlength{\textheight}{604pt}
\renewcommand{\footrulewidth}{0.1mm}
\fancyhf{}
        \fancyhead[LO]{\bf \includegraphics[width=80pt]{eseo.png}\\
        						\medskip
                                 {\prose} équipe {\teamNumber} {\annee}}
         \fancyhead[RO]{\bf \includegraphics[width=40pt]{logo_entreprise.png}\\
				\medskip
				{\small{Ref. {\documentNameAbrev}\_E{\teamNumber}}}}
         \fancyfoot[LO]{\sl {\it Version {\version} - Révision {\revision}}}
         \cfoot{\copyright {\annee} Droits réservés {\teamName}}
         \fancyfoot[RO]{\thepage/{\completer}}

\setcounter{tocdepth}{3}
%------------------------------------


%----------------------------------------
%             Parameters for adding a subsubsubsection, command and 
%             make it appear in the tableofcontent
%             It also modify the paragraph ans subparagraph command
%----------------------------------------

\titleclass{\subsubsubsection}{straight}[\subsection]
\newcounter{subsubsubsection}[subsubsection]
\renewcommand\thesubsubsubsection{\thesubsubsection.\arabic{subsubsubsection}}
\renewcommand\theparagraph{\thesubsubsubsection.\arabic{paragraph}} % optional; useful if paragraphs are to be numbered
\titleformat{\subsubsubsection}
  {\normalfont\normalsize\bfseries}{\thesubsubsubsection}{1em}{}
\titlespacing*{\subsubsubsection}
{0pt}{3.25ex plus 1ex minus .2ex}{1.5ex plus .2ex}
\makeatletter
\renewcommand\paragraph{\@startsection{paragraph}{5}{\z@}%
  {3.25ex \@plus1ex \@minus.2ex}%
  {-1em}%
  {\normalfont\normalsize\bfseries}}
\renewcommand\subparagraph{\@startsection{subparagraph}{6}{\parindent}%
  {3.25ex \@plus1ex \@minus .2ex}%
  {-1em}%
  {\normalfont\normalsize\bfseries}}
\def\toclevel@subsubsubsection{4}
\def\toclevel@paragraph{5}
\def\toclevel@paragraph{6}
\def\l@subsubsubsection{\@dottedtocline{4}{7em}{4em}}
\def\l@paragraph{\@dottedtocline{5}{10em}{5em}}
\def\l@subparagraph{\@dottedtocline{6}{14em}{6em}}
\makeatother
\setcounter{secnumdepth}{4}
\setcounter{tocdepth}{4}
%------------------------------------


%----------------------------------------
%       DOCUMENT
%----------------------------------------

\begin{document}

%---------------------------------------

\sloppy%
% Modifie l’espacement vertical entre les lignes d’un tableau (tabular)
\renewcommand{\arraystretch}{1.1}

%---------------------------------------

%\vspace{-2cm}%
\begin{center}%
\vspace*{1cm}
\rule[0.5ex]{0.4\textwidth}{0.1mm}\\
\vspace*{2mm}
{\Huge {\textsc{\bf {\documentName}}}}
\vspace{0.4cm}\\
{\large\bf {\prose} {\teamNumber} {\annee} - {\teamName}}\\
{\large\bf {\projectName}}

\rule[0.5ex]{0.4\textwidth}{0.1mm}

\vspace{1cm}

\end{center}
\begin{center}
\begin{tabular}{|c|c|}
\hline
Responsable du document & Xxxxxx Xxxxxx                \\
État du document        & En cours de création         \\
Version                 & {\version}  \\
Révision                & {\revision} \\
\hline
\end{tabular}
\end{center}

\vspace{2cm}
\noindent
\textbf{AVERTISSEMENT :}\\
Le présent document est un document à but pédagogique. 
Il a été réalisé sous la direction de Jérôme Delatour, 
en collaboration avec des enseignants 
et des étudiants de l'option SE du groupe ESEO. 
Ce document a été traduit en \LaTeX par Clément Le Goffic.\\
Ce document est la propriété de Jérôme Delatour du groupe ESEO. 
En dehors des activités pédagogiques de l'ESEO, ce document ne peut 
être diffusé ou recopié sans l’autorisation écrite de son propriétaire (Jérôme Delatour). %Comment if not needed

\newpage

% Auteur : Camille Constant

\noindent
% Modifie l’espacement horizontal entre les colonnes
%\setlength{\tabcolsep}{5pt}
\begin{tabularx}{\linewidth}{|p{1.8cm}|X|p{2cm}|p{1.5cm}|p{1.5cm}|}
\hline
\textbf{Date} & \textbf{Actions} & \textbf{Auteur} & \textbf{Version} & \textbf{Révision}  \\
\hline
11/02/2022 & Création du document & {\creatorAbrev} & 0.0 & 0\\
11/02/2022 & Appropriation du document & {\creatorAbrev} & 0.0 & 1\\
\hline
\end{tabularx}

\newpage

%------------------------------- 

\tableofcontents
% alternative pour réduire l'espacement entre les entrées de la table des matières
% (la valeur numérique peut être adaptée au besoin) : 
%{\setlength{\baselineskip}{0.96\baselineskip}\tableofcontents\par}
\newpage

%-------------------------------
% Ajouter toutes les parties les unes après les autres, séparées par un \newpage
% Exemple : 
% \input{Introduction}
% \newpage
\section{But}%#1
\subsection{Objectifs du document}%#1.1
Ce document est un Plan d’Assurance Qualité Logicielle (PAQL) visant à définir toutes les règles, 
les méthodes et les outils utilisés dans le projet {\projectName} afin de définir et 
contrôler la qualité du projet.\\
Ce document poursuit les objectifs suivants :
\begin{itemize}
  \item Définir le niveau de qualité attendu par l’équipe projet pour le projet ProSE.
  \item Définir les outils utilisés, les processus et procédures à suivre par l’équipe 
  projet tant au niveau organisationnel que technique lors du projet ProSE.
\end{itemize}
Ce document est disponible sur le Référentiel Documentaire Projet(RDP) dans le répertoire 
Qualité\slash Redaction\_PAQL sous le nom "{\documentName}".

\subsection{Portée}%#1.2
Ce document est destiné à :
\begin{itemize}
  \item À l'équipe projet,
  \item Aux consultants de la société FORMATO
\end{itemize}

\subsection{Copyright}%#1.3
\noindent
\textbf{AVERTISSEMENT :}\\
Le présent document est un document à but pédagogique. 
Il a été réalisé sous la direction de Jérôme Delatour, 
en collaboration avec des enseignants 
et des étudiants de l'option SE du groupe ESEO. 
Ce document a été traduit en \LaTeX par Clément Le Goffic.\\
Ce document est la propriété de Jérôme Delatour du groupe ESEO. 
En dehors des activités pédagogiques de l'ESEO, ce document ne peut 
être diffusé ou recopié sans l’autorisation écrite de son propriétaire (Jérôme Delatour).


\subsection{Vue d’ensemble}%#1.4
Ce PAQL est structuré suivant les grandes parties proposées par la norme [IEEE\-730\_1998].\\
Il est donc décomposé en 24 parties.\\
La norme IEEE 730 décrit le contenu d'un plan d'AQL pour un logiciel:
\begin{itemize}
  \item Intention et Portée
  \item Définitions et Abréviations
  \item Documents de références
  \item Survol du plan d'assurance qualité Logicielle
  \begin{itemize}
    \item Organisation
    \item Niveau de criticité du logiciel
    \item Outils, techniques et méthodologies
    \item Ressources
    \item Normes, pratiques et conventions
    \item Calendriers
  \end{itemize}
  \item Activités et tâches de cycle de vis de l'AQL
  \begin{itemize}
    \item Rôle de l'assurance de produit
    \item Rôle de l'assurance du processus
    \item Assurances sur les activités et les tâches du système de management de la qualité
    \item Activités et tâches additionnelles
  \end{itemize}
  \item Processus et politiques additionnelles
  \begin{itemize}
    \item Processus de revue de contrat
    \item Processus de mesure de la qualité
    \item Politiques de tests
    \item Politique de dérogation et de déviation
    \item Politique d'itération des tâches
  \end{itemize}
  \item Enregistrements et rapports de l'AQL
  \begin{itemize}
    \item Enregistrements
    \item Rapports
  \end{itemize}
\end{itemize}

\subsection{Références}%#1.5

\nocite{*}
\bibliographystyle{plain}
\bibliography{biblio.bib}

\section{Gestion}%#2
Cette partie décrit l’organisation, les tâches et les responsabilités 
en rapport avec les activités d’Assurance Qualité (AQ) du projet {\projectName}.

\subsection{Organisation}%#2.1

\subsubsection{Projet concerné}%#2.1.1
Le projet \completer\\
Voir dossier de spécification.


\subsubsection{Ressources humaines}%#2.1.2

\subsubsubsection{Équipe Projet}%#2.1.2.1
\noindent
% Modifie l’espacement horizontal entre les colonnes
%\setlength{\tabcolsep}{5pt}
\begin{tabularx}{\linewidth}{|X|X|X|p{5cm}|X|}
\hline
\textbf{Rôle} & \textbf{Nom} & \textbf{Prénom} & \textbf{Mail} & \textbf{Téléphone}  \\
\hline
Responsable Qualité Test & Le Goffic & Clément & clement.legoffic@reseau.eseo.fr & +33 7 82 77 51 25\\
\hline
\end{tabularx}

\subsubsubsection{Client}%#2.1.2.2
Le Client est \completer

\subsubsubsection{Consultants et auditeurs}%#2.1.2.3
Si nécessaire, l'équipe projet pourra faire appel à la 
société FORMATO en tant que support technique. 
Les consultants et leurs compétences privilégiées sont :
\begin{itemize}
  \item Jérôme DELATOUR (spécification / conception / qualité / gestion de projet) : jerome.delatour@eseo.fr
  \item Matthias BRUN (codage android / tests) : matthias.brun@eseo.fr
  \item Camille CONSTANT (qualité / tests) : camille.constant@eseo.fr
  \item Frédéric JOUAULT (codage C / conception) : frederic.jouault@eseo.fr
\end{itemize}
Des activités d’audits externes (cf. chapitre 5.3, page {\completer}) 
seront menées par les auditeurs de la société FORMATO ou missionnées par elle.

\subsection{Tâches du projet}%#2.2
\subsubsection{Tâches transversales}%#2.2.1
Les tâches transversales de l’assurance qualité incluent les activités suivantes :
\begin{itemize}
  \item Documentation (cf. chapitre 3, page 22).
  \item Revues et audits (cf. chapitre 5 , page 31 )
  \item Inspections internes
  \item Validation et tests
  \item Activités d’amélioration du processus d’AQ
\end{itemize}

\subsubsubsection{Inspections internes}%#2.2.1.1
\completer
\\
(Les activités à mener pous assurer que la qualité est tenue, audits réguliers tenue redmine
ex qui va relire le PAQL)

\subsubsubsection{Validation et test}%#2.2.1.2
% Le Plan de test a pour objectif d’identifier les informations existantes du projet 
% et les composants qui doivent être testés. Il énumère les exigences d’évaluation à 
% différents niveaux, décrit les stratégies de test qui seront employées, identifie 
% les ressources nécessaires et met en évidence les biens livrables pour les tests.\\
\completer (voir plan de test)
% \begin{itemize}
%   \item Portée du document, termes et abréviations
%   \item Références
%   \item Périmètre de test (composants concernés ou non par les tests, 
%   fonctionnalités testées ou non, critères d'acceptation des tests)
%   \item Processus et stratégie de test(activités, techniques, outils, 
%   procédures de test et gestion des anomalies)
%   \item Infrastructure de test
%   \item Documents de test et livrables
%   \item Responsabilités
%   \item Équipe de test
%   \item Planning prévisionnel
% \end{itemize}
Cf : [Plan de test \completer]

\subsubsubsection{Évolution et amélioration du PAQL}
Le PAQL est susceptible d'évoluer au cours du projet, en particulier pour les raisons suivantes :
\begin{itemize}
  \item Toutes les informations nécessaires à la rédaction d'un chapitre 
  ou d'un paragraphe ne sont pas connues ou suffisamment stabilisées lors de la rédaction.
  \item Il s'agit d'une phase du cycle de développement qui sera engagée ultérieurement 
  (cas de la mention « Rédaction réservée »).
  \item Des événements techniques ou organisationnels nécessitant une prise en 
  compte dans le PAQL peuvent apparaître lors du déroulement du projet (modification 
  d'organisation, mise en place de nouvelles normes ou de procédures ou modification 
  de normes ou procédures existantes, ...).
\end{itemize}
Le PAQL est rédigé par le Responsable Qualité et Test (RQT) de l’équipe projet. 
Le Chef de Projet (CdP) et les RQT participent aux décisions de modifications. 
Il incombe au RQT d’effectuer les modifications jugées nécessaires du PAQL. 
En cas de modifications du PAQL, celui-ci devra être signé à nouveau par les 
membres de l’équipe projet.

\subsubsection{Tâches par rapport au cycle de développement}
L’équipe projet suivra un cycle de développement en V en deux incréments. 
Les activités d’AQ sont décrites par rapport à ce cycle. 
Le planning et les échéances associées sont disponibles sur 
l’Espace Numérique de Travail du Projet (ENTP).
\begin{figure}[H]
  \centering
  \includegraphics[width=15cm]{cycle.png}
  \caption{Diagramme du cycle de développement du projet}
\end{figure}

\subsubsubsection{Phase d'initialisation du projet}
\begin{table}[H]
  \renewcommand{\arraystretch}{1.1}
  \begin{tabular}{|lll|}
  \hline
  \rowcolor[HTML]{CCCCCC} 
  \multicolumn{3}{|l|}{\cellcolor[HTML]{CCCCCC}\textbf{Phase : Initialisation}}                                                                                                                                                                                                                                                                   \\ \hline
  \multicolumn{3}{|l|}{
    \begin{tabular}[c]{@{}l@{}}
      Objectifs :\\ 
      Prendre en charge le projet, l’organiser, le planifier et en valider les bases.\\ 
      Évaluer les actions nécessaires pour mettre en place le projet.\\ 
      Échanger avec l'équipe sur les règles à définir.
    \end{tabular}}                                                                                                                                                                                                                                 \\ \hline
  \multicolumn{3}{|l|}{
    \begin{tabular}[c]{@{}l@{}}
      Remarques :\\
    \end{tabular}}                                                                                                                                                                                                                                                                \\ \hline
  \multicolumn{1}{|l|}{
    \begin{tabular}[c]{@{}l@{}}
      Acteurs \& reponsabilités :\\ 
      \tabitem CdP et RQT
    \end{tabular}}&\multicolumn{1}{l|}{
    \begin{tabular}[c]{@{}l@{}}
      Méthodes \& Règles :\\ 
      \tabitem Règles pour l'utilisation\\ de l'ENTP\\
      \tabitem Anticipation et organisation\\des deadlines personnelles
    \end{tabular}}& 
    \begin{tabular}[c]{@{}l@{}}
      Moyens \& Outils :\\ 
      \tabitem Initialisation du projet\\ sous ENTP
    \end{tabular}\\ \hline
  \multicolumn{1}{|l|}{
    \begin{tabular}[c]{@{}l@{}}
      Activités \\ d'organisation/pilotage :\\ 
      \tabitem Organisation de la \\réunion de lancement\\ 
      \tabitem Organisation de la phase\\ en aval
    \end{tabular}}& \multicolumn{1}{l|}{
      \begin{tabular}[c]{@{}l@{}}
        Activités de \\ production/soutien :\\ 
        \tabitem Élaboration PAQL\\ 
        \tabitem Mise en place de l’ENTP\\
        \tabitem Définition de la démarche\\ du projet\\
        \tabitem Initialisation du planning \\et du suivi du projet
      \end{tabular}}& 
      \begin{tabular}[c]{@{}l@{}}
        Activités de \\ vérification/contrôle :\\ 
        \tabitem Réunion de lancement\\
        \tabitem \completer
      \end{tabular}\\ \hline
  \multicolumn{1}{|l|}{
    \begin{tabular}[c]{@{}l@{}}
      Produits/données en entrée :\\ 
      \tabitem Wiki Prose et RedMine\\ 
      \tabitem Documents pédagogiques\\
    \end{tabular}}& \multicolumn{1}{l|}{
      \begin{tabular}[c]{@{}l@{}}
        Produits/données en sortie :\\ 
        \tabitem Planning des tâches \\sur l'ENTP\\ 
      \end{tabular}}           & 
      \begin{tabular}[c]{@{}l@{}}
        Produits révisés :\\ 
        \tabitem PAQL
      \end{tabular}\\ \hline
  \multicolumn{3}{|l|}{\begin{tabular}[c]{@{}l@{}}
    Jalons de la phase :\\ 
    \tabitem J1 : \completer
  \end{tabular}}                                                                                                                                                                                                                                                    \\ \hline
  \rowcolor[HTML]{CCCCCC} 
  \multicolumn{1}{|l|}{\cellcolor[HTML]{CCCCCC}
  \begin{tabular}[c]{@{}l@{}}
    Conditions \\ de début de phase :
  \end{tabular}} & \multicolumn{1}{l|}{\cellcolor[HTML]{CCCCCC}
  \begin{tabular}[c]{@{}l@{}}
    Condition \\ de fin de phase :
  \end{tabular}} & 
  \begin{tabular}[c]{@{}l@{}}
    Conditions \\ de passage à la \\phase suivante :
  \end{tabular}\\ \hline
  \multicolumn{1}{|l|}{
    \begin{tabular}[c]{@{}l@{}}
      \tabitem Nomination des CdP et \\RQT
    \end{tabular}}& \multicolumn{1}{l|}{
      \begin{tabular}[c]{@{}l@{}}
        \tabitem Validation des futurs \\livrables\\
        \tabitem Validation de l'équipe \\du planning prévisionnel
      \end{tabular}}& 
      \begin{tabular}[c]{@{}l@{}}
        \tabitem ENTP opérationnel\\
        \tabitem Attribution des rôles
      \end{tabular}\\ \hline
  \end{tabular}
\end{table}

\subsubsubsection{Phase de spécification}
Spécifications : Le dossier de spécification devra respecter le plan défini par la norme 
\cite[IEEE\-830\_1998]{830} et s’appuyer sur la notation UML \cite[UML\_2.4\_2011]{UML}. 
Deux audits (un consultatif et un normatif) porteront sur le dossier de 
spécification. Le plan de test ainsi que le cahier de test de validation 
seront établis durant cette étape de spécification. Deux audits 
(un consultatif et un normatif) porteront sur cette activité. Une revue de 
mi-avancement aura lieu pour présenter au client le dossier de spécification 
et les éléments contractuels.

\begin{table}[H]
  \renewcommand{\arraystretch}{1.1}
  \begin{tabular}{|lll|}
  \hline
  \rowcolor[HTML]{CCCCCC} 
  \multicolumn{3}{|l|}{\cellcolor[HTML]{CCCCCC}\textbf{PHASE : SPECIFICATION V1 \& V2}}                                                                                                                                                                                                                                                                   \\ \hline
  \multicolumn{3}{|l|}{
    \begin{tabular}[c]{@{}l@{}}
      Objectifs :\\ 
      Mener des activités d’exploration techniques afin d’évaluer la complexité et le temps \\
      nécessaire à la réalisation du futur produit.\\ 
      Présenter les principales fonctions, les performances requises, les exigences de \\
      qualité et les contraintes de réalisation.\\ 
      Faire la description complète de toutes les fonctionnalités des sous-ensembles du projet.\\
	  Présenter le dossier final de spécification.\\
	  Présenter le plan de test.\\
	  Lancer les explorations techniques nécessaires au projet.
    \end{tabular}}                                                                                                                                                                                                                                 \\ \hline
  \multicolumn{3}{|l|}{
    \begin{tabular}[c]{@{}l@{}}
      Remarques :\\
    \end{tabular}}                                                                                                                                                                                                                                                                \\ \hline
  \multicolumn{1}{|l|}{
    \begin{tabular}[c]{@{}l@{}}
      Acteurs \& reponsabilités :\\ 
      \tabitem CdP et équipe\\
	  \tabitem Client
    \end{tabular}}&\multicolumn{1}{l|}{
    \begin{tabular}[c]{@{}l@{}}
      Méthodes \& Règles :\\ 
      \tabitem PAQL
    \end{tabular}}& 
    \begin{tabular}[c]{@{}l@{}}
      Moyens \& Outils :\\ 
      \tabitem ENTP
    \end{tabular}\\ \hline
  \multicolumn{1}{|l|}{
    \begin{tabular}[c]{@{}l@{}}
      Activités \\ d'organisation/pilotage :\\ 
      \tabitem Organiser les échanges \\d’informations avec le client 
    \end{tabular}}& \multicolumn{1}{l|}{
      \begin{tabular}[c]{@{}l@{}}
        Activités de \\ production/soutien :\\ 
        \tabitem Rédaction du dossier\\ de spécifications\\ 
        \tabitem Rédaction du plan\\ de test\\
        \tabitem Élaboration des \\maquettes des écrans\\
        \tabitem Explorations technique\\
		\tabitem \completer
      \end{tabular}}& 
      \begin{tabular}[c]{@{}l@{}}
        Activités de \\ vérification/contrôle :\\ 
        \tabitem AC sur spécification\\
        \tabitem AN sur spécification (date)\\
        \tabitem \completer
      \end{tabular}\\ \hline
  \multicolumn{1}{|l|}{
    \begin{tabular}[c]{@{}l@{}}
      Produits/données en entrée :\\ 
      \tabitem Cahier des charges Client\\ 
      \tabitem PAQL\\
    \end{tabular}}& \multicolumn{1}{l|}{
      \begin{tabular}[c]{@{}l@{}}
        Produits/données en sortie :\\ 
        \tabitem Dossier de spécifications\\
		\tabitem Plan de test\\
		\tabitem Contrat client\\
      \end{tabular}}           & 
      \begin{tabular}[c]{@{}l@{}}
        Produits révisés :\\ 
        \tabitem ENTP\\
		    \tabitem Planning prévisionnel\\
		    \tabitem Plan de test\\
		    \tabitem PAQL
      \end{tabular}\\ \hline
  \multicolumn{3}{|l|}{\begin{tabular}[c]{@{}l@{}}
    Jalons de la phase :\\ 
    \tabitem J1 : \completer \\
	\tabitem \\
	\tabitem \\
  \end{tabular}}                                                                                                                                                                                                                                                    \\ \hline
  \rowcolor[HTML]{CCCCCC} 
  \multicolumn{1}{|l|}{\cellcolor[HTML]{CCCCCC}
  \begin{tabular}[c]{@{}l@{}}
    Conditions \\ de début de phase :
  \end{tabular}} & \multicolumn{1}{l|}{\cellcolor[HTML]{CCCCCC}
  \begin{tabular}[c]{@{}l@{}}
    Condition \\ de fin de phase :
  \end{tabular}} & 
  \begin{tabular}[c]{@{}l@{}}
    Conditions \\ de passage à la \\phase suivante :
  \end{tabular}\\ \hline
  \multicolumn{1}{|l|}{
    \begin{tabular}[c]{@{}l@{}}
      \tabitem Initialisation effectuée
    \end{tabular}}& \multicolumn{1}{l|}{
    \begin{tabular}[c]{@{}l@{}}
      \tabitem Validation des produits\\ en sortie
      \tabitem Signature du client\\
    \end{tabular}}& 
    \begin{tabular}[c]{@{}l@{}}
      \tabitem Dossier de spécifications \\signé par le client\\
      \tabitem Plan de test validé\\
    \end{tabular}\\ \hline
  \end{tabular}
\end{table}

\subsubsubsection{Phase de conception}
Conception générale (ou système) : Un audit consultatif 
portera sur la conception générale.\\
Conception détaillée : Un audit consultatif portera sur la conception détaillée.\\
Un audit normatif portera sur la conception générale et détaillée.

\begin{table}[H]
  \renewcommand{\arraystretch}{1.1}
  \begin{tabular}{|lll|}
  \hline
  \rowcolor[HTML]{CCCCCC} 
  \multicolumn{3}{|l|}{\cellcolor[HTML]{CCCCCC}\textbf{PHASE : Conception V1 \& V2}}                                                                                                                                                                                                                                                                   \\ \hline
  \multicolumn{3}{|l|}{
    \begin{tabular}[c]{@{}l@{}}
      Objectifs :\\ 
      Organiser et optimiser les temps de conception.\\ 
      Définir l’architecture logicielle des applications.\\ 
      Finaliser les ultimes explorations techniques et les intégrer au dossier.\\
	  Décrire explicitement le dossier de conception.\\
    \end{tabular}}                                                                                                                                                                                                                                 \\ \hline
  \multicolumn{3}{|l|}{
    \begin{tabular}[c]{@{}l@{}}
      Remarques :\\
	  \tabitem La conception générale peut commencer en parallèle des spécifications.
    \end{tabular}}                                                                                                                                                                                                                                                                \\ \hline
  \multicolumn{1}{|l|}{
    \begin{tabular}[c]{@{}l@{}}
      Acteurs \& reponsabilités :\\ 
      \tabitem CdP et équipe\\
	  \tabitem Client
    \end{tabular}}&\multicolumn{1}{l|}{
    \begin{tabular}[c]{@{}l@{}}
      Méthodes \& Règles :\\ 
      \tabitem PAQL\\
	  \tabitem Dossier de spécification\\
    \end{tabular}}& 
    \begin{tabular}[c]{@{}l@{}}
      Moyens \& Outils :\\ 
      \tabitem ENTP
    \end{tabular}\\ \hline
  \multicolumn{1}{|l|}{
    \begin{tabular}[c]{@{}l@{}}
      Activités \\ d'organisation/pilotage :\\ 
      \tabitem Organiser les échanges \\d’informations avec le client 
    \end{tabular}}& \multicolumn{1}{l|}{
      \begin{tabular}[c]{@{}l@{}}
        Activités de \\ production/soutien :\\ 
        \tabitem Définition de l’architecture\\ technique\\
        \tabitem Conception support de \\communication \\Rasberry/IPX800V4\\
        \tabitem Conception logicielle\\
        \tabitem Normalisation des fonctions\\
		\tabitem Rédaction du dossier de \\conception\\
		\tabitem Rédaction des tests de \\validation\\
		\tabitem \completer
      \end{tabular}}& 
      \begin{tabular}[c]{@{}l@{}}
        Activités de \\ vérification/contrôle :\\ 
        \tabitem \completer
      \end{tabular}\\ \hline
  \multicolumn{1}{|l|}{
    \begin{tabular}[c]{@{}l@{}}
      Produits/données en entrée :\\ 
      \tabitem Dossier de spécifications\\ 
      \tabitem PAQL\\
    \end{tabular}}& \multicolumn{1}{l|}{
      \begin{tabular}[c]{@{}l@{}}
        Produits/données en sortie :\\ 
        \tabitem Dossier de conception\\
      \end{tabular}}           & 
      \begin{tabular}[c]{@{}l@{}}
        Produits révisés :\\ 
        \tabitem Normes de développement\\
		\tabitem Dossier de spécifications\\
		\tabitem PAQL
      \end{tabular}\\ \hline
  \multicolumn{3}{|l|}{\begin{tabular}[c]{@{}l@{}}
    Jalons de la phase :\\ 
    \tabitem J1 : \completer \\
  \end{tabular}}                                                                                                                                                                                                                                                    \\ \hline
  \rowcolor[HTML]{CCCCCC} 
  \multicolumn{1}{|l|}{\cellcolor[HTML]{CCCCCC}
  \begin{tabular}[c]{@{}l@{}}
    Conditions \\ de début de phase :
  \end{tabular}} & \multicolumn{1}{l|}{\cellcolor[HTML]{CCCCCC}
  \begin{tabular}[c]{@{}l@{}}
    Condition \\ de fin de phase :
  \end{tabular}} & 
  \begin{tabular}[c]{@{}l@{}}
    Conditions \\ de passage à la \\phase suivante :
  \end{tabular}\\ \hline
  \multicolumn{1}{|l|}{
    \begin{tabular}[c]{@{}l@{}}
      \tabitem \completer
    \end{tabular}}& \multicolumn{1}{l|}{
      \begin{tabular}[c]{@{}l@{}}
        \tabitem Validation des produits\\ en sortie
      \end{tabular}}& 
      \begin{tabular}[c]{@{}l@{}}
        \tabitem Dossier de conception validé\\
        \tabitem Normes de développement\\
		\tabitem Tests de validation\\ rédigés
      \end{tabular}\\ \hline
  \end{tabular}
\end{table}

\subsubsubsection{Phase de réalisation}
Codage : Quatre audits porteront sur le code source produit afin notamment 
de s’assurer du bon respect du PAQL et des normes de programmation. 
Deux porteront sur le code écrit en langage C sur cible embarquée 
(consultatif et normatif) et deux autres sur le code fonctionnant sur la plate-forme Android.

\begin{table}[H]
  \renewcommand{\arraystretch}{1.1}
  \begin{tabular}{|lll|}
  \hline
  \rowcolor[HTML]{CCCCCC} 
  \multicolumn{3}{|l|}{\cellcolor[HTML]{CCCCCC}\textbf{PHASE : Réalisation V1 \& V2}}                                                                                                                                                                                                                                                                   \\ \hline
  \multicolumn{3}{|l|}{
    \begin{tabular}[c]{@{}l@{}}
      Objectifs :\\ 
      Développer des applications logicielles.\\ 
      Tester les applications logicielles définies lors de la phase de conception.\\ 
      Coder les tests d'intégration et unitaires et adopter les outils de test.\\
    \end{tabular}}                                                                                                                                                                                                                                 \\ \hline
  \multicolumn{3}{|l|}{
    \begin{tabular}[c]{@{}l@{}}
      Remarques :\\
	  \tabitem L’étape de réalisation peut commencer en parallèle de la conception.
    \end{tabular}}                                                                                                                                                                                                                                                                \\ \hline
  \multicolumn{1}{|l|}{
    \begin{tabular}[c]{@{}l@{}}
      Acteurs \& reponsabilités :\\ 
      \tabitem CdP et équipe\\
    \end{tabular}}&\multicolumn{1}{l|}{
    \begin{tabular}[c]{@{}l@{}}
      Méthodes \& Règles :\\ 
      \tabitem PAQL\\
	  \tabitem Dossier de spécification\\
	  \tabitem Dossier de conception\\
    \end{tabular}}& 
    \begin{tabular}[c]{@{}l@{}}
      Moyens \& Outils :\\ 
      \tabitem ENTP\\
	    \tabitem Moyen de test\\
      \tabitem Testlink \& Eclipse\\ 
    \end{tabular}\\ \hline
  \multicolumn{1}{|l|}{
    \begin{tabular}[c]{@{}l@{}}
      Activités \\ d'organisation/pilotage :\\ 
      \tabitem Organiser les fonctions \\prioritaires\\
	  \tabitem Organiser les échanges \\d'informations avec le client
    \end{tabular}}& \multicolumn{1}{l|}{
      \begin{tabular}[c]{@{}l@{}}
        Activités de \\ production/soutien :\\ 
		\tabitem \completer
      \end{tabular}}& 
      \begin{tabular}[c]{@{}l@{}}
        Activités de \\ vérification/contrôle :\\ 
        \tabitem \completer
      \end{tabular}\\ \hline
  \multicolumn{1}{|l|}{
    \begin{tabular}[c]{@{}l@{}}
      Produits/données en entrée :\\ 
      \tabitem Dossier de conception\\ 
      \tabitem PAQL\\
    \end{tabular}}& \multicolumn{1}{l|}{
      \begin{tabular}[c]{@{}l@{}}
        Produits/données en sortie :\\ 
        \tabitem Code\\
		\tabitem Moyen de test opérationnel\\
		\tabitem tests unitaires\\
		\tabitem \completer
      \end{tabular}}           & 
      \begin{tabular}[c]{@{}l@{}}
        Produits révisés :\\ 
        \tabitem Normes de développement\\
		\tabitem Dossier de spécifications\\
		\tabitem Dossier de conception\\
		\tabitem PAQL
      \end{tabular}\\ \hline
  \multicolumn{3}{|l|}{\begin{tabular}[c]{@{}l@{}}
    Jalons de la phase :\\ 
    \tabitem J1 : \completer \\
  \end{tabular}}                                                                                                                                                                                                                                                    \\ \hline
  \rowcolor[HTML]{CCCCCC} 
  \multicolumn{1}{|l|}{\cellcolor[HTML]{CCCCCC}
  \begin{tabular}[c]{@{}l@{}}
    Conditions \\ de début de phase :
  \end{tabular}} & \multicolumn{1}{l|}{\cellcolor[HTML]{CCCCCC}
  \begin{tabular}[c]{@{}l@{}}
    Condition \\ de fin de phase :
  \end{tabular}} & 
  \begin{tabular}[c]{@{}l@{}}
    Conditions \\ de passage à la \\phase suivante :
  \end{tabular}\\ \hline
  \multicolumn{1}{|l|}{
    \begin{tabular}[c]{@{}l@{}}
      \tabitem \completer
    \end{tabular}}& \multicolumn{1}{l|}{
      \begin{tabular}[c]{@{}l@{}}
        \tabitem Validation des produits\\ en sortie
      \end{tabular}}& 
      \begin{tabular}[c]{@{}l@{}}
        \tabitem Application logicielle \\réalisée\\
		\tabitem Tests réalisables
      \end{tabular}\\ \hline
  \end{tabular}
\end{table}

\subsubsubsection{Phase de test}
Tests unitaires, tests d’intégration : Un audit consultatif 
portera sur chacune de ces activités.\\
Tests de validation : Un audit normatif portera sur 
l’application des tests (validation, intégration et unitaire).\\

\begin{table}[H]
  \renewcommand{\arraystretch}{1.1}
  \begin{tabular}{|lll|}
  \hline
  \rowcolor[HTML]{CCCCCC} 
  \multicolumn{3}{|l|}{\cellcolor[HTML]{CCCCCC}\textbf{PHASE : Test V1 \& V2}}                                                                                                                                                                                                                                                                   \\ \hline
  \multicolumn{3}{|l|}{
    \begin{tabular}[c]{@{}l@{}}
      Objectifs :\\ 
      Développer des tests fiables et efficaces.\\ 
      Contrôler la fiabilité du logiciel.\\ 
      Identifier les erreurs logiques.\\
	  Vérifier interactions des interfaces.\\
	  Valider l'adéquation aux spécifications du logiciel.\\
    \end{tabular}}                                                                                                                                                                                                                                 \\ \hline
  \multicolumn{3}{|l|}{
    \begin{tabular}[c]{@{}l@{}}
      Remarques :\\
	  \tabitem La phase de programmation des tests peut commencer en parallèle de la réalisation
    \end{tabular}}                                                                                                                                                                                                                                                                \\ \hline
  \multicolumn{1}{|l|}{
    \begin{tabular}[c]{@{}l@{}}
      Acteurs \& reponsabilités :\\ 
      \tabitem CdP et Resp Test\\
	  \tabitem Équipe de test
    \end{tabular}}&\multicolumn{1}{l|}{
    \begin{tabular}[c]{@{}l@{}}
      Méthodes \& Règles :\\ 
      \tabitem PAQL\\
	  \tabitem Conception\\
	  \tabitem Spécification\\
	  \tabitem Plan de test\\
	  \tabitem Cahier de tests
    \end{tabular}}& 
    \begin{tabular}[c]{@{}l@{}}
      Moyens \& Outils :\\ 
      \tabitem ENTP\\
	    \tabitem Moyen de test\\
      \tabitem Testlink
    \end{tabular}\\ \hline
  \multicolumn{1}{|l|}{
    \begin{tabular}[c]{@{}l@{}}
      Activités \\ d'organisation/pilotage :\\ 
      \tabitem Vérification croisée\\
	  \tabitem Organiser les tests \\principaux\\
	  \tabitem \completer
    \end{tabular}}& \multicolumn{1}{l|}{
      \begin{tabular}[c]{@{}l@{}}
        Activités de \\ production/soutien :\\ 
		\tabitem Exécution tests de \\communication\\
		\tabitem Compléter cahier de test\\
		\tabitem Exécution  tests robustesse\\
		\tabitem Exécution  tests unitaires\\
		\tabitem Exécution  tests validation\\
      \end{tabular}}& 
      \begin{tabular}[c]{@{}l@{}}
        Activités de \\ vérification/contrôle :\\ 
        \tabitem \completer
      \end{tabular}\\ \hline
  \multicolumn{1}{|l|}{
    \begin{tabular}[c]{@{}l@{}}
      Produits/données en entrée :\\ 
      \tabitem Application logicielle\\ 
      \tabitem Dossier de conception\\
	  \tabitem PAQL\\
	  \tabitem Moyen de test\\
	  \tabitem Cahier de test\\
    \end{tabular}}& \multicolumn{1}{l|}{
      \begin{tabular}[c]{@{}l@{}}
        Produits/données en sortie :\\ 
        \tabitem Document d'analyse des tests\\
		\tabitem Application logicielle testée\\
		\tabitem Cahier de test\\
		\tabitem \completer
      \end{tabular}}           & 
      \begin{tabular}[c]{@{}l@{}}
        Produits révisés :\\ 
        \tabitem Normes de développement\\
		\tabitem Document de conception\\
		\tabitem Document de spécifications\\
		\tabitem Code
      \end{tabular}\\ \hline
  \multicolumn{3}{|l|}{\begin{tabular}[c]{@{}l@{}}
    Jalons de la phase :\\ 
    \tabitem J1 : \completer \\
  \end{tabular}}                                                                                                                                                                                                                                                    \\ \hline
  \rowcolor[HTML]{CCCCCC} 
  \multicolumn{1}{|l|}{\cellcolor[HTML]{CCCCCC}
  \begin{tabular}[c]{@{}l@{}}
    Conditions \\ de début de phase :
  \end{tabular}} & \multicolumn{1}{l|}{\cellcolor[HTML]{CCCCCC}
  \begin{tabular}[c]{@{}l@{}}
    Condition \\ de fin de phase :
  \end{tabular}} & 
  \begin{tabular}[c]{@{}l@{}}
    Conditions \\ de passage à la \\phase suivante :
  \end{tabular}\\ \hline
  \multicolumn{1}{|l|}{
    \begin{tabular}[c]{@{}l@{}}
      \tabitem Plan de test \\partiellement validé
    \end{tabular}}& \multicolumn{1}{l|}{
    \begin{tabular}[c]{@{}l@{}}
      \tabitem Validation des produits\\ en sortie
    \end{tabular}}& 
    \begin{tabular}[c]{@{}l@{}}
      \tabitem Tests réalisés\\
      \tabitem Cahier de test complet
    \end{tabular}\\ \hline
  \end{tabular}
\end{table}

\subsubsubsection{Phase de recette}
Une revue de recette aura lieu pour remettre au client le produit demandé, 
ainsi que fournir les livrables suivants : les dossiers de spécification 
et conception, l'ensemble des tests réalisés (cahier et plan de test), 
les codes sources, les manuels d'utilisation et d'installation, le procès 
verbal de recette définitive ainsi que l'application logicielle fonctionnelle.

\begin{table}[H]
  \renewcommand{\arraystretch}{1.1}
  \begin{tabular}{|lll|}
  \hline
  \rowcolor[HTML]{CCCCCC} 
  \multicolumn{3}{|l|}{\cellcolor[HTML]{CCCCCC}\textbf{PHASE : Recette}}                                                                                                                                                                                                                                                                   \\ \hline
  \multicolumn{3}{|l|}{
    \begin{tabular}[c]{@{}l@{}}
      Objectifs :\\ 
      Réceptionner les applications de manière définitive\\
    \end{tabular}}                                                                                                                                                                                                                                 \\ \hline
  \multicolumn{3}{|l|}{
    \begin{tabular}[c]{@{}l@{}}
      Remarques :\\
	  \tabitem Se fera en même temps que la phase de bilan de fin de projet
    \end{tabular}}                                                                                                                                                                                                                                                                \\ \hline
  \multicolumn{1}{|l|}{
    \begin{tabular}[c]{@{}l@{}}
      Acteurs \& reponsabilités :\\ 
      \tabitem CdP et équipe\\
	  \tabitem Client
    \end{tabular}}&\multicolumn{1}{l|}{
    \begin{tabular}[c]{@{}l@{}}
      Méthodes \& Règles :\\ 
      \tabitem PAQL\\
	  \tabitem Conception\\
	  \tabitem Spécification\\
	  \tabitem Intégration
    \end{tabular}}& 
    \begin{tabular}[c]{@{}l@{}}
      Moyens \& Outils :\\ 
      \tabitem ENTP
    \end{tabular}\\ \hline
  \multicolumn{1}{|l|}{
    \begin{tabular}[c]{@{}l@{}}
      Activités \\ d'organisation/pilotage :\\ 
      \tabitem Organiser les échanges \\d’informations avec le client
    \end{tabular}}& \multicolumn{1}{l|}{
      \begin{tabular}[c]{@{}l@{}}
        Activités de \\ production/soutien :\\ 
		\tabitem \completer		
      \end{tabular}}& 
      \begin{tabular}[c]{@{}l@{}}
        Activités de \\ vérification/contrôle :\\ 
        \tabitem \completer
      \end{tabular}\\ \hline
  \multicolumn{1}{|l|}{
    \begin{tabular}[c]{@{}l@{}}
      Produits/données en entrée :\\ 
      \tabitem PAQL\\
	  \tabitem Dossier de specification\\
	  \tabitem Dossier de conception\\
	  \tabitem Cahier de test\\
	  \tabitem Code
    \end{tabular}}& \multicolumn{1}{l|}{
      \begin{tabular}[c]{@{}l@{}}
        Produits/données en sortie :\\ 
        \tabitem Application opérationnelle \\livrée\\
		\tabitem Manuel d’utilisation et \\manuel d’installation validés\\
		\tabitem Dernière version à jour \\des documents, codes \\source des applications \\(documenté sous Doxygen) \\et exécutables\\
		\tabitem Procès verbal de recette \\définitive
      \end{tabular}}           & 
      \begin{tabular}[c]{@{}l@{}}
        Produits révisés :\\ 
        \tabitem Cahier de test\\
		\tabitem Manuel d’utilisation\\
		\tabitem Manuel d’installation\\
		\tabitem Code source\\
		\tabitem Recette\\
		\tabitem \completer
      \end{tabular}\\ \hline
  \multicolumn{3}{|l|}{\begin{tabular}[c]{@{}l@{}}
    Jalons de la phase :\\ 
    \tabitem J1 : \completer \\
  \end{tabular}}                                                                                                                                                                                                                                                    \\ \hline
  \rowcolor[HTML]{CCCCCC} 
  \multicolumn{1}{|l|}{\cellcolor[HTML]{CCCCCC}
  \begin{tabular}[c]{@{}l@{}}
    Conditions \\ de début de phase :
  \end{tabular}} & \multicolumn{1}{l|}{\cellcolor[HTML]{CCCCCC}
  \begin{tabular}[c]{@{}l@{}}
    Condition \\ de fin de phase :
  \end{tabular}} & 
  \begin{tabular}[c]{@{}l@{}}
    Conditions \\ de passage à la \\phase suivante :
  \end{tabular}\\ \hline
  \multicolumn{1}{|l|}{
    \begin{tabular}[c]{@{}l@{}}
      \tabitem Phase d’intégration terminée
    \end{tabular}}& \multicolumn{1}{l|}{
    \begin{tabular}[c]{@{}l@{}}
      \tabitem Validation des produits\\ en sortie
    \end{tabular}}& 
    \begin{tabular}[c]{@{}l@{}}
      \tabitem Fin du projet\\
    \end{tabular}\\ \hline
  \end{tabular}
\end{table}

\subsubsubsection{Phase de bilan de fin de projet}

\begin{table}[H]
  \renewcommand{\arraystretch}{1.1}
  \begin{tabular}{|lll|}
  \hline
  \rowcolor[HTML]{CCCCCC} 
  \multicolumn{3}{|l|}{\cellcolor[HTML]{CCCCCC}\textbf{PHASE : Bilan Fin de Projet}}\\ \hline
  \multicolumn{3}{|l|}{
    \begin{tabular}[c]{@{}l@{}}
      Objectifs :\\ 
      Clôturer le projet, archiver les productions, effectuez les statistiques du projet (temps passé), \\
      tirer les enseignements du projet.\\
    \end{tabular}}\\ \hline
  \multicolumn{3}{|l|}{
    \begin{tabular}[c]{@{}l@{}}
      Remarques :\\
	  \tabitem Se fera en même temps que la phase de recette.
    \end{tabular}}\\ \hline
  \multicolumn{1}{|l|}{
    \begin{tabular}[c]{@{}l@{}}
      Acteurs \& reponsabilités :\\ 
      \tabitem CdP\\
    \end{tabular}}&\multicolumn{1}{l|}{
    \begin{tabular}[c]{@{}l@{}}
      Méthodes \& Règles :\\ 
      \tabitem PAQL\\
    \end{tabular}}& 
    \begin{tabular}[c]{@{}l@{}}
      Moyens \& Outils :\\ 
      \tabitem ENTP\\
    \end{tabular}\\ \hline
  \multicolumn{1}{|l|}{
    \begin{tabular}[c]{@{}l@{}}
      Activités \\ d'organisation/pilotage :\\ 
      \tabitem Organiser les échanges \\d’informations avec le référent \\ESEO et M. Delatour
    \end{tabular}}& \multicolumn{1}{l|}{
      \begin{tabular}[c]{@{}l@{}}
        Activités de \\ production/soutien :\\ 
		\tabitem Rédaction des documents\\ de bilan\\	
      \end{tabular}}& 
      \begin{tabular}[c]{@{}l@{}}
        Activités de \\ vérification/contrôle :\\ 
        \tabitem Audit Interne\\
		\tabitem \completer
      \end{tabular}\\ \hline
  \multicolumn{1}{|l|}{
    \begin{tabular}[c]{@{}l@{}}
      Produits/données en entrée :\\ 
      \tabitem ENTP\\
	  \tabitem Livrables client\\
	  \tabitem RDP\\
    \end{tabular}}& \multicolumn{1}{l|}{
      \begin{tabular}[c]{@{}l@{}}
        Produits/données en sortie :\\ 
        \tabitem Statistiques projet\\
		    \tabitem Compte-rendu de réunion\\ de bilan de projet\\
		    \tabitem Synthèse enseignements projets\\
		    \tabitem Propositions amélioration\\
      \end{tabular}}           & 
      \begin{tabular}[c]{@{}l@{}}
        Produits révisés :\\ 
        \tabitem ENTP\\
		\tabitem Livrables\\
		\tabitem RDP\\
      \end{tabular}\\ \hline
  \multicolumn{3}{|l|}{\begin{tabular}[c]{@{}l@{}}
    Jalons de la phase :\\ 
    \tabitem J1 : \completer \\
  \end{tabular}}      \\ \hline
  \rowcolor[HTML]{CCCCCC} 
  \multicolumn{1}{|l|}{\cellcolor[HTML]{CCCCCC}
  \begin{tabular}[c]{@{}l@{}}
    Conditions \\ de début de phase :
  \end{tabular}} & \multicolumn{1}{l|}{\cellcolor[HTML]{CCCCCC}
  \begin{tabular}[c]{@{}l@{}}
    Condition \\ de fin de phase :
  \end{tabular}} & 
  \begin{tabular}[c]{@{}l@{}}
    Conditions \\ de passage à la \\phase suivante :
  \end{tabular}\\ \hline
  \multicolumn{1}{|l|}{
    \begin{tabular}[c]{@{}l@{}}
      \tabitem Phase d’intégration terminée
    \end{tabular}}& \multicolumn{1}{l|}{
    \begin{tabular}[c]{@{}l@{}}
      \tabitem Validation des produits\\ en sortie
    \end{tabular}}& 
    \begin{tabular}[c]{@{}l@{}}
      \tabitem Fin du projet\\
    \end{tabular}\\ \hline
  \end{tabular}
\end{table}

\subsection{Responsabilité}
\subsubsection{Définition Générale des rôles}

Chaque membre de l’équipe projet est tenu de respecter et d’appliquer les 
normes du PAQL dans son travail.\\
Chaque membre pourra avoir l’un des rôles suivants :
\begin{itemize}
  \item \textbf{Chef de projet (CdP) :} \completer
  \item \textbf{Responsable Qualité et Test (RQT) :} \completer
  \item \textbf{Développeur :} \completer
  \item \textbf{Responsable Développeur C | Java (RD) :} \completer
\end{itemize}

\subsubsection{Récapitulatif des responsabilités client sur les phases}

\begin{table}[H]
  \begin{tabular}{|l|l|l|}
  \hline
  \rowcolor[HTML]{CCCCCC} 
  \textbf{Phase} &
    \textbf{Implication Client} &
    \textbf{Implication Équipe Projet} \\ \hline
  \textbf{Initialisation} &
    \completer &
    \tabitem \completer \\ \hline
  \textbf{Spécification} &
    \begin{tabular}[c]{@{}l@{}}
      \tabitem Répondre aux questions \\concernant l'étude des besoins\\ 
      \tabitem Valider l'IHM\\ 
      \tabitem Fournir l'existant système \\si nécessaire\\ 
      \tabitem Valider les dossiers
    \end{tabular} 
      &
    \begin{tabular}[c]{@{}l@{}}
      \tabitem Réaliser l'étude des besoins\\ 
      \tabitem Rédiger le dossier de spécifications\\ 
      \tabitem Rédiger le cahier de recette
    \end{tabular} \\ \hline
  \textbf{Conception} &
    \tabitem \completer &
    \begin{tabular}[c]{@{}l@{}}
      \tabitem Rédiger le dossier de conception\\ 
      \tabitem Rédiger le cahier de test
    \end{tabular} \\ \hline
  \textbf{Réalisation} &
    \tabitem \completer &
    \begin{tabular}[c]{@{}l@{}}
      \tabitem Développer les modules logiciels\\ 
      \tabitem Tests unitaires de l'application
    \end{tabular} \\ \hline
  \textbf{Test} &
    \tabitem \completer &
    \begin{tabular}[c]{@{}l@{}}
      \tabitem Analyse des tests\\ 
      \tabitem Vérification finale de l'application
    \end{tabular} \\ \hline
  \textbf{Recette} &
    \begin{tabular}[c]{@{}l@{}}
      \tabitem Valider le taux de \\fiabilité des tests\\ 
      \tabitem Valider le fonctionnement \\global de l'application\\ 
      \tabitem Valider l'ensemble des \\livrables reçus\\ 
      \tabitem Signature du procès-verbal \\de recette définitive
    \end{tabular} &
    \begin{tabular}[c]{@{}l@{}}
      \tabitem Rédaction du procès-verbal de recette\\ 
      \tabitem Livrer l'ensemble des livrables
    \end{tabular} \\ \hline
  \textbf{\begin{tabular}[c]{@{}l@{}}Bilan de fin\\ de projet\end{tabular}} &
    \tabitem \completer &
    \begin{tabular}[c]{@{}l@{}}
      \tabitem Faire une analyse et \\les statistiques du projet\\
      \tabitem Proposer des améliorations\\ 
      \tabitem Rédiger un rapport de fin de projet
    \end{tabular} \\ \hline
  \end{tabular}
  \end{table}

\subsubsection{Récapitulatif des responsabilités CdP sur les phases}

\begin{table}[H]
  \begin{tabular}{|l|l|}
  \rowcolor[HTML]{CCCCCC} 
  \hline
  \textbf{Phase}          & \textbf{Implication CdP}                       \\ \hline
  \textbf{Initialisation} & \ti{\tabitem \completer\\ \tabitem \completer} \\ \hline
  \textbf{Spécification}  & \ti{\tabitem \completer\\ \tabitem \completer} \\ \hline
  \textbf{Conception}     & \ti{\tabitem \completer\\ \tabitem \completer} \\ \hline
  \textbf{Réalisation}    & \ti{\tabitem \completer\\ \tabitem \completer} \\ \hline
  \textbf{Test}           & \ti{\tabitem \completer\\ \tabitem \completer} \\ \hline
  \textbf{Recette}        & \ti{\tabitem \completer\\ \tabitem \completer} \\ \hline
  \textbf{\ti{Bilan de fin\\ de projet}} & \ti{\tabitem \completer\\ \tabitem \completer} \\ \hline
  \end{tabular}
\end{table}

\subsubsection{Récapitulatif des responsabilités RQ sur les phases}

\begin{table}[H]
  \begin{tabular}{|l|l|}
  \rowcolor[HTML]{CCCCCC}
  \hline 
  \textbf{Phase}          & \textbf{Implication RQT}                       \\ \hline
  \textbf{Initialisation} & \ti{\tabitem \completer\\ \tabitem \completer} \\ \hline
  \textbf{Spécification}  & \ti{\tabitem \completer\\ \tabitem \completer} \\ \hline
  \textbf{Conception}     & \ti{\tabitem \completer\\ \tabitem \completer} \\ \hline
  \textbf{Réalisation}    & \ti{\tabitem \completer\\ \tabitem \completer} \\ \hline
  \textbf{Test}           & \ti{\tabitem \completer\\ \tabitem \completer} \\ \hline
  \textbf{Recette}        & \ti{\tabitem \completer\\ \tabitem \completer} \\ \hline
  \textbf{\ti{Bilan de fin\\ de projet}} & \ti{\tabitem \completer\\ \tabitem \completer} \\ \hline
  \end{tabular}
\end{table}

\subsubsection{Récapitulatif des responsabilités des développeurs sur les phases}

\begin{table}[H]
  \begin{tabular}{|l|l|}
  \rowcolor[HTML]{CCCCCC} 
  \hline
  \textbf{Phase}          & \textbf{Implication Équipe de développement}   \\ \hline
  \textbf{Initialisation} & \ti{\tabitem \completer\\ \tabitem \completer} \\ \hline
  \textbf{Spécification}  & \ti{\tabitem \completer\\ \tabitem \completer} \\ \hline
  \textbf{Conception}     & \ti{\tabitem \completer\\ \tabitem \completer} \\ \hline
  \textbf{Réalisation}    & \ti{\tabitem \completer\\ \tabitem \completer} \\ \hline
  \textbf{Test}           & \ti{\tabitem \completer\\ \tabitem \completer} \\ \hline
  \textbf{Recette}        & \ti{\tabitem \completer\\ \tabitem \completer} \\ \hline
  \textbf{\ti{Bilan de fin\\ de projet}} & \ti{\tabitem \completer\\ \tabitem \completer} \\ \hline
  \end{tabular}
\end{table}

\section{Documentation}
\subsection{But}
Ce chapitre décrit les règles de gestion de la documentation du projet. 
En effet, un certain nombre d’artefacts du projet concerne des documents.

\subsection{Type de documents}
Les documents suivants sont distingués suivant leur nature, 
qu’ils soient livrés par le client ou non, consultables par 
les auditeurs ou réservés à l’équipe projet, et ce suivant 
la phase où ils sont produits.\\
Les artefacts de documentation, nommés documents "livrables client" sont :
\begin{itemize}
  \item Dossier de spécification
  \item Plan d'Assurance Qualité Logicielle (PAQL)
  \item Contrat et devis
  \item Dossier de conception
  \item Plan de tests
  \item Artefacts de code
  \item Cahier de tests
  \item Manuel d'utilisation et manuel d'installation
  \item Présentation revue
\end{itemize}
Les artefacts consultables par les consultants, 
nommés documents "consultables auditeur", sont :
\begin{itemize}
  \item Tous les documents "livrables client"
  \item Ordre du jour réunion projet
  \item Compte-rendu réunion client
  \item Compte-rendu réunion projet
  \item Correspondance échangée avec le client
  \item Les codes sources des explorations techniques
  \item Les codes sources
  \item Les codes de test
\end{itemize}
Les autres documents du projet sont considérés comme internes au projet.

\subsection{Référence des documents}
La référence d’un document est de la forme suivante : 
« SIGLE\_EXX », où XX désigne l’identifiant de l’équipe 
(une lettre et un numéro) et le SIGLE correspond à 
l’une des combinaisons de lettres citées ci-dessous.\\

Ce système de référencement ne sera appliqué que pour 
les livrables "consultables auditeur". Les ébauches et 
documents internes à l'équipe échappent donc 
à cette règle de nommage.\\

\begin{table}[H]
  \begin{tabular}{|l|l|}
  \hline
  \rowcolor[HTML]{CCCCCC} 
  \textbf{Réference}           & \textbf{Libellé Document}           \\ \hline
  PAQL\_{\teamNumber}          & Plan d'Assurance Qualité Logicielle \\ \hline
  SPEC\_{\teamNumber}          & Document de spécification           \\ \hline
  CONC\_{\teamNumber}          & Document de conception              \\ \hline
  TEST\_{\teamNumber}          & Document de test                    \\ \hline
  BILAN\_{\teamNumber}         & Bilan de projet                     \\ \hline
  \completer                   & \completer                          \\ \hline
  \completer                   & \completer                          \\ \hline
  \completer                   & \completer                          \\ \hline
  \completer                   & \completer                          \\ \hline
  \completer                   & \completer                          \\ \hline
  SANS\_E{\teamNumber}          & Pour tous les autres documents      \\ \hline
  \end{tabular}
\end{table}
\tabitem \completer\\
Par exemple, pour le dossier de spécification, produit par l’équipe XX, 
sa référence sera la suivante : « SPEC\_EXX ».\\
La référence d’un document permet d’identifier la catégorie à laquelle 
appartient ce document, cette référence figurera dans le 
document, mais aussi dans le nom de son fichier.\\
Ainsi, dans l’exemple précédent, le fichier 
pourra se nommer "dossier\_de\_specification\_SPEC\_EXX".


\subsection{État d’un document}
\completer\\
Les différents états d'un document sont :
\begin{itemize}
  \item \textbf{En création :} Document en cours de création
  \item \textbf{En construction :} Document en cours de développement
  \item \textbf{En relecture :} Docuement en attente de relecture
  \item \textbf{En validation :} Docuement en attente de validation
  \item \textbf{Version finale :} Document validé et livrable
\end{itemize}

\begin{figure}[H]
  \centering
  \includegraphics[width=15cm]{etatsDoc.png}
  \caption{Etats d'un document au cours du temps}
\end{figure}

\subsection{Responsable du document}
Un seul membre d’une équipe est responsable d’un document. 
Tout document doit avoir un responsable désigné. C’est ce responsable qui 
suit l’évolution du document et ses différents états.\\

C’est aussi le responsable qui gérera les cas où différents 
membres d’une équipe désirent travailler en même temps sur la 
même version d’un document dont il est le responsable. 
Il lui appartient alors de mettre en œuvre la stratégie 
qui lui semble la mieux adaptée, soit en :
\begin{itemize}
  \item Séquençant les mises à jour
  \item Demandant à ce que les membres lui envoient leur version 
  afin qu’il les intègre lui-même dans le document et qu’il 
  le pose alors dans le RDP. (Dans ce dernier cas, chacun 
  des contributeurs pose sa contribution sur le RDP avant intégration.)
\end{itemize}
\completer

\subsection{Processus d’édition d’un document}
\completer\\

Le processus d'édition de document nous permet d'assurer un suivi en 
continu sur les différents documents du projet et ainsi être 
en mesure de réagir rapidement en cas de pertes d'informations 
ou pour retrouver l'origine d'erreurs commises.
\begin{figure}[H]
  \centering
  \includegraphics[width=15cm]{processEdition.png}
  \caption{Processus d'édition d'un document}
\end{figure}
Déterminer les flux sur le donc qui quoi quand ?

\subsection{version d’un document}
\completer\\
Dans le PAQL ici présenté, il s’agit d’une version, sur 2 nombres X.Y.\\
X s’incrémente à chaque validation d’un document (X s'incrémentera donc 
au maximum jusqu'à 3). Y s'incrémente à chaque modification publiée du 
document (chaque fois qu'une personne termine une phase de rédaction et 
publie le document sur le RDP).\\

Pensez à mettre à jour le numéro de version dans le tableau en tête de 
document, ainsi que sur la page de couverture et sur les pieds de page. 
L'indication du numéro de révision \footnote{Le numéro de révision s'incrémente 
automatiquement à chaque enregistrement du document.} 
n'est pas nécessaire sur les pieds 
de page des documents mais il convient de reporter le numéro de révision 
en vis-à-vis de la version correspondante dans ce même tableau.\\

\subsection{Format des documents}
Pour tous les documents "livrables client" et "consultables auditeur", 
il faut respecter un modèle de documents. Ces modèles 
sont disponibles sur le RDP, dans le dossier Qualité.

\subsubsection{Modèle de document}
Des modèles de documents sont proposés suivant le sigle de ces documents. 
Ces modèles doivent être impérativement utilisés. 
Ils sont disponibles dans le RDP, dans le dossier « Qualite/Modeles ». 
Tous les fichiers seront au format OpenDocument (modifiables et 
lisibles avec LibreOffice). Veillez à activer le suivi des modifications 
pour tous ces fichiers.

\subsubsection{Artefact de code}
\completer
Dans le RDP, dossier « Qualite/Modeles », 
les modèles d'artefact de code suivants sont disponibles :
\begin{itemize}
  \item example.h
  \item example.c
  \item example.java
\end{itemize}
Il est impératif que tous les codes source produits dans le projet 
utilisent ces modèles et respectent les conventions de nommage qu’ils 
proposent (détaillées ci-après). Il est impératif que les codes source 
soient parfaitement commentés et ce en utilisant les balises au format 
doxygen. Ce point est crucial, car il permet la génération automatique 
de la documentation du code source de tout le projet.\\
Avec ces fichiers modèles, vous trouverez également des 
fichiers utilitaires dont l’utilisation est vivement recommandée.\\
Afin de préciser certaines règles de codage qui sont difficiles à 
illustrer dans les modèles précédemment cités, voici quelques exemples :
\subsubsection{Règles de codage en langage C :}
\completer

\subsubsection{Règles de traduction de la conception vers du code en langage C}
\completer\\
Les règles de traduction de la conception vers du code C devront 
impérativement suivre celles vues en cours avec les auditeurs ProSE.\\
Cf : [Cours\_Programmation\_Avancé\_2016]

\subsubsection{Règles de codage en langage Java}
\completer\\

\subsubsection{Règles de traduction de la conception vers du code en Java}
\completer\\
Les règles de traduction de la conception vers du code Java 
devront impérativement suivre celles vues en cours avec les auditeurs ProSE.\\
Cf : [Cours\_Programmation\_Avancé\_2016]

\subsection{Documents internes}
Tout document interne, n'étant pas considéré comme document livrable, 
n'est pas soumis aux conventions de nommage ni de mise en forme. 
Ces documents internes ne passeront pas par le processus d'édition de 
document cité plus haut (pas de relecture ni de validation). Il est 
entièrement de la responsabilité du responsable du document de vérifier 
son contenu et expliciter synthétiquement son objectif dans son nommage 
et sur l'ENTP.
\section{Standards, pratiques, conventions et métriques}
\subsection{But}
Cette section décrit les standards, pratiques, conventions et métriques 
utilisés pour le projet ProSE.  Ceux-ci ont pour but d’assurer la 
qualité du logiciel tout en fournissant des données quantitatives 
sur le processus d’AQ.
\subsection{Exigences qualités générales}

\begin{table}[H]
  \begin{tabular}{|ll|}
  \hline
  \multicolumn{2}{|l|}{\cellcolor[HTML]{CCCCCC}Description des exigences qualité} \\ \hline
  \multicolumn{1}{|l|}{\begin{tabular}[c]{@{}l@{}}Liées au produit\\ (par ordre décroissant de priorité)\end{tabular}} &
    \begin{tabular}[c]{@{}l@{}}1.    Conformité : le produit livré devra être conforme au dossier de spécification livré et livré dans les délais promis.\\ 2.    Maintenabilité (Aptitude du produit à permettre une maintenance facile, rapide et peu coûteuse).\\ 3.    Adaptabilité (Aptitude de la partie logicielle à supprimer ou modifier les fonctionnalités existantes, ou ajouter de nouvelles fonctionnalités).\\ 4.    Maniabilité (Aptitude du produit à être convivial et facile d’emploi pour l’utilisateur).\end{tabular} \\ \hline
  \multicolumn{1}{|l|}{\begin{tabular}[c]{@{}l@{}}Liées au processus\\ (par ordre décroissant de priorité)\end{tabular}} &
    \begin{tabular}[c]{@{}l@{}}1.    Traçabilité.\\ 2.    Conformité (au présent PAQL et normes indiquées).\\ 3.    Simplicité.\end{tabular} \\ \hline
  \end{tabular}
\end{table}


\subsection{Exigences qualités sur les artefacts}
Pour tous les artefacts remis au client ou consultés par les auditeurs, 
une gestion de version et un suivi des modifications devront être activés 
afin de permettre la fourniture de n’importe quelle version d’un artefact 
de ce type et de pouvoir identifier clairement les modifications apportées 
entre 2 versions.
\subsubsection{Exigences sur les documents consultables par les auditeurs}
Concernant les documents "consultables auditeur", tous devront respecter 
la même présentation et respecteront les mêmes modèles de document 
disponibles dans le dossier /qualite/modeles du RDP.

\subsubsection{Exigences sur les documents livrables}
Tous les documents livrables, avant d’être remis au client devront avoir 
été relus et corrigés. Hormis pour les artefacts de code, une version pdf 
doit être disponible pour les livrables lors de leur livraison.\\
Critère de qualité sur les documents livrables :
\begin{itemize}
  \item Respect du modèle de document et ses champs.
  \item Pas plus de deux fautes d’orthographe par page du document.
\end{itemize}
Le document de spécification devra se baser sur la norme IEEE 830 \cite[IEEE\-830\_1998]{830} 
et utiliser la notation UML \cite[UML\_2.4\_2011]{UML}. 
Le document de conception devra utiliser la notation UML \cite[UML\_2.4\_2011]{UML}.
\subsubsection{Exigences sur le code source}

Concernant les codes sources du projet, ils devront respecter les règles de programmation 
et les conventions de nommage associées. Afin de permettre une meilleure maintenabilité 
et lisibilité, ils seront documentés en utilisant des balises doxygen.\\
Le code source livré devra être compilable et pouvoir produire un exécutable fonctionnel.\\
\completer
Il y aura au minimum les cinq balises suivantes dans les artefacts de code :
\begin{itemize}
  \item @file : le nom du fichier
  \item @brief : le résumé du contenu du fichier
  \item @version : la version du fichier
  \item @date : la date de création du fichier
  \item @author : le créateur du fichier et développeurs impliqués
  \item @copyright : le lien vers la licence BSD
\end{itemize}

\section{Revues et Audits}

\subsection{But}
Cette section présente les actions d’audit interne, externe et de revue qui 
pourront être menées afin d’évaluer la qualité du projet, et ce sur 
différentes activités.
\subsection{Revues}
\subsubsection{Revue de mi-avancement}
La revue de mi-avancement permet de présenter l’ensemble des actions menées 
sur le premier incrément du cycle en V aux acteurs externes au projet. 
Un ensemble de détails est présenté sur le Wiki ProSE à l’adresse : 
[WikiProSE\_Exigences\_Pedagogiques\_SE]. Tous les membres de l’équipe 
doivent être présents et intervenir durant la présentation. L’équipe disposera 
de 20 minutes de présentation et de 5 minutes de démonstration, suivies de 
25 minutes de questions.
\subsubsection{Revue de recette}
De même que pour la revue de mi-avancement, les précisions ont été faites 
sur le site du Wiki ProSE à l'adresse [WikiProSE\_Exigences\_Pedagogiques\_SE]. 
Tous les membres de l’équipe doivent être présents et intervenir durant la 
présentation. L’équipe disposera de 20 minutes de présentation, dont 5 minutes 
de démonstrations, suivies de 30 minutes de questions.
\subsection{Audits}
Des audits externes seront menés par les consultants FORMATO durant toute 
la vie du projet. Un planning prévisionnel des audits est donné sur l’ENTP. 
Toutefois, les dates d’audit ne sont données qu’à titre indicatif, les 
auditeurs pouvant décaler leur audit d’une ou deux séances suivant leurs 
disponibilités. L’ordre de passage des équipes des audits au cours d’une 
séance n’est d’ailleurs jamais connu et reste à la discrétion de l’auditeur. 
L’équipe est tenue de mettre à disposition un membre compétent lors des 
audits externes.\\
Il y a 2 types d'audits externes :
\begin{itemize}
  \item Les audits consultatifs
  \item Les audits normatifs
\end{itemize}
Lors de chacun de ces audits, l'équipe devra fournir un membre ayant le rôle 
de secrétaire qui prendra en notes les remarques et les changements à 
effectuer qui seront cités lors de l'audit. Le compte-rendu de cet audit 
sera déposé ensuite sur le RDP dans le dossier /gestion\_projet/audit sous 
le nom :  "A(C ou N) JJ\-MM\-AAAA\_SANS\_EXX".

\subsubsection{Audit consultatif}
Les consultants vous donneront une indication sur le travail que vous avez 
fait sous forme de code couleur (grade) sur l’avancement de votre projet. 
La signification des grades est la suivante :
\begin{itemize}
  \item Le grade OR est attribué à des travaux de qualité exemplaire 
  \item Le grade VERT est attribué à des travaux satisfaisants (de corrects à très bons)
  \item Le grade ORANGE est attribué à des travaux présentant quelques lacunes mais ne portant pas de préjudice grave pour la suite du projet 
  \item Le grade ROUGE est attribué à des travaux déficients, présentant des lacunes importantes et dommageables pour le projet.
\end{itemize}
L’audit consultatif n’est pas noté et n’est pas transmis au client.

\subsubsection{Audit normatif}
Cet audit nécessite une présentation soignée des documents audités. Il
faut à minima que les remarques faites lors de l'audit consultatif 
correspondant aient été prises en compte et corrigées. Un code couleur 
(le même que pour un audit consultatif) est attribué.\\

Le rapport d’audit normatif est transmis au client et sera pris 
en compte pour l'évaluation finale du projet.

\subsubsection{Inspection et revue croisée}

Pour les inspections internes, il convient de respecter les 
règles énoncées dans le chapitre 3 \completer.

\section{Test}
\completer

\section{Notification des problèmes et corrections}
\completer
La notification des problèmes, à toute étape du processus de 
développement, se fait directement sur l’ENTP et ce par chacun 
des membres de l’équipe. Cela se fait par l’émission d’une demande 
de type « Bug » sur l’outil Redmine de l’ENTP.\\
Les états possibles d'une demande Bug sont :
\begin{itemize}
  \item Nouveau
  \item En cours
  \item Fermé
  \item Réouvert
  \item Résolu
\end{itemize}
L'émission d'une telle demande se fera selon le formalisme 
suivant qui permet de tracer les origines des dysfonctionnements :
\begin{figure}[H]
  \centering
  \includegraphics[width=15cm]{MAEBug.png}
  \caption{Machine à état du processus de traitement d'un dysfonctionnement}
\end{figure}
Lorsqu’un membre de l'équipe signale un bug, l'état du bug devient [\textbf{Nouveau}], 
si le bug à déjà été déclaré, son état sera [\textbf{Réouvert}] , puis si le Bug est 
pris en charge par un membre, l'état passe à [\textbf{En cours}]. Si la correction 
arrive à terme, l'état suivant sera [\textbf{Résolu}]. Dans le cas où la correction 
est impossible, due aux coûts, aux délais ou à des moyens techniques insuffisants, 
l'état deviendra [\textbf{Non résolu}].\\

C'est le chef de projet qui décidera de l'assignation des membres aux bugs qui 
seront détectés. Le diagramme suivant permet d'illustrer l'activité des membres 
impliqués lors d'une tâche de Debug :
\begin{figure}[H]
  \centering
  \includegraphics[width=15cm]{ActiBug.png}
  \caption{Diagramme d'activité expliquant le processus de traitement d'un dysfonctionnement}
\end{figure}

\section{Outils, Techniques et Methodologie}
\subsection{L’espace Numérique de Travail du Projet (ENTP)}
L’ENTP est un espace vous offrant les outils suivants :
\begin{itemize}
  \item Redmine : un gestionnaire de suivi de projet.
  \item WikiProse : Wiki contenant un ensemble d’informations sur le projet ProSE, Wiki proposé par l’équipe pédagogique.
  \item Le RDP (Réfentiel Document Projet) : C'est le dépôt subversion.
\end{itemize}
L'ENTP est accessible de la même façon à l'intérieur et à l'extérieur de l'ESEO.\\
Les identifiants (login et mot de passe), lorsqu'ils sont réclamés, 
correspondent à ceux créés sur Redmine au début du projet.\\
Résumé des URL pour les différents composants de l'ENTP :

\begin{table}[H]
  \begin{tabular}{|l|l|}
  \hline
  \rowcolor[HTML]{CCCCCC} 
  Composant       & URL                               \\ \hline
  WikiProSE       & http://prose.eseo.fr/prose/       \\ \hline
  Redmine         & http://prose.eseo.fr/redmine/     \\ \hline
  Tableau de bord & http://prose.eseo.fr/prose-board/ \\ \hline
  RDP             & \completer                        \\ \hline
  \end{tabular}
\end{table}

\subsubsection{Redmine}
Redmine est un gestionnaire de projet ayant comme fonctionnalités :
\begin{itemize}
  \item La gestion de plusieurs projets paramétrables
  \item La gestion des utilisateurs
  \item La gestion de documents
  \item La gestion de demandes
  \item Les priorités paramétrables d'une demande
  \item Un historique
  \item Un historique
  \item La modulation fine des statuts et la gestion des transitions de statuts par rôle.
  \item L'ajout de champs personnalisés
  \item La gestion du temps
\end{itemize}
N’oubliez pas de vérifier tous les jours la page d’accueil, dans la 
rubrique "Dernières Annonces". Vous pourrez y trouvez des 
informations venant des consultants (ex : salle de formation, horaires, etc.).

\subsubsection{Planning prévisionnel}
\completer
La version détaillée du planning prévisionnel est à saisir sur Redmine, 
par le biais de demandes de type « Version ». Ces demandes sont à 
hiérarchiser pour offrir une lecture du plus global au plus détaillé. 
Par exemple :
\begin{itemize}
  \item Spécifications
  \item \begin{itemize}
    \item Cas d'utilisation
    \item Cas d'utilisation stratégique
    \item IHM
    \item Rédaction du dossier
  \end{itemize}
\end{itemize}
Ces phases servent à structurer le planning prévisionnel du projet. 
A ce titre, les informations à saisir sont généralement 
(à adapter selon les situations) :
\begin{itemize}
  \item titre et description
  \item date de début et échéance
  \item relation de dépendances avec d'autres phases projet (précède, suit, …)
  \item saisie de temps
\end{itemize}
Les activités proprement dites sont suivies par le biais de demandes 
de type « tâche ». Ces demandes sont nécessairement des sous-demandes 
de demandes de type « Version ». Les demandes de type « tâche » :
\begin{itemize}
  \item ne doivent pas être hiérarchisées par rapport aux membres
  \item doivent être affectées à une et une seule personne pour laquelle elle représente le travail à faire
  \item sont dupliquées autant que nécessaire si plusieurs personnes effectuent le même travail
\end{itemize}
Pour le bon fonctionnement et une bonne gestion du projet, seul le chef de projet est 
autorisé à modifier le planning prévisionnel. Le CdP est donc le seul responsable de 
la concordance dans le temps du planning prévisionnel. Néanmoins, les autres membres 
de l'équipe sont autorisés à modifier, ajouter/supprimer ou modifier une tâche tout 
en respectant la structuration de tâches imposée par le CdP, et ceci pour garder une 
cohérence dans l'ensemble des tâches réalisées au cours du projet.

\subsubsection{Suivi du travail}
\subsubsubsection{Une mise à jour immédiate sur le RDP}
À chaque fois qu’un membre effectue un travail (en séance ou hors séance), 
il doit saisir le temps qu’il y a consacré sur Redmine par le biais de 
ses demandes de type "tâche" et compléter l’état d’avancement de la 
demande. De plus, il devra alors déposer la nouvelle version de ces 
artefacts projet sur le RDP.\\
\underline{Remarque :} La seule preuve du temps consacré à une tâche est le dépôt 
des artefacts. Un temps saisi sur Redmine sans dépôt associé peut ne 
pas être pris en compte par le CdP ou les consultants lors des audits 
"projet".\\
De plus, une réunion d'équipe projet est organisée de façon hebdomadaire.\\
Cette réunion a pour but de faire un point sur le travail de la semaine 
effectuée et de tenir à jour l'avancée du projet et ainsi établir les objectifs 
de la semaine à venir. Avant chacune de ces réunions, un ordre du jour est 
envoyé à chacun des membres de l'équipe à minima 48h avant le début de la 
réunion et chacun doit le lire et en prendre connaissance avant la réunion.
Lors de chacune de ces réunions, un secrétaire est désigné pour rapporter 
les éléments importants abordés et les publier sous forme de compte-rendu 
sur le RDP dans le dossier gestion de projet.

\subsubsubsection{Synthèse Personnelle}
Une synthèse du temps passé (dite synthèse personnelle) devra être 
régulièrement mis à jour sur le wiki de l'ENTP. Il sera demandé de 
remplir définitivement et impérativement la synthèse personnelle 
avant les audits individuels.\\
Voici un exemple de synthèse personnelle :\\
\\
Prénom NOM\\
Nombres totales d'heures travaillées :\\
Spécifications => 29 h\\
\begin{itemize}
  \item IHM => 12h ; Définition des écrans Android et MAE
  \item Rédaction Introduction de spécifications v3.0 [seul] => 2h
  \item Relecture des CU  => 6h : CU Emprunter Objet, 
  Rendre Objet, Manipuler Posto, Administrateur,
  \item Mise en cohérence des CU entre eux => 6h
  \item Travail en équipe sur l’ensemble des CU => 3h
\end{itemize}
Conception => 40h, responsable de la V1
\begin{itemize}
  \item CRC : 8h, en équipe avec Prénom Nom, ....
  \item Diagramme de séquences : 8h : Emprunter Objet, connecter
  \item MAE : 8h : MAE de l'IHM
  \item \dots
\end{itemize}
Et ainsi de suite pour les activités de test, de gestion de la qualité, 
de codage, de gestion de projet (pas à détailler) et de consulting (pas à détailler)...\\
\completer

\subsection{Liste des outils autorisés}
Voici la liste de tous les outils que l’équipe projet 
est autorisée à utiliser lors du projet.\\ 
\completer Et adapter (conseil logiciel libre)

\begin{table}[H]
  \begin{tabular}{|l|l|}
  \hline
  \rowcolor[HTML]{CCCCCC} 
  \textbf{Usage}                  & \textbf{Outil utilisé}                                                    \\ \hline
  Traitement de texte             & \begin{tabular}[c]{@{}l@{}}\completer\\ Version : \completer\end{tabular} \\ \hline
  Tableur                         & \begin{tabular}[c]{@{}l@{}}\completer\\ Version : \completer\end{tabular} \\ \hline
  Diagramme UML                   & \begin{tabular}[c]{@{}l@{}}\completer\\ Version : \completer\end{tabular} \\ \hline
  Gestion de test                 & \begin{tabular}[c]{@{}l@{}}\completer\\ Version : \completer\end{tabular} \\ \hline
  Présentation/Soutenance de projet & \begin{tabular}[c]{@{}l@{}}\completer\\ Version : \completer\end{tabular} \\ \hline
  Gestion de version              & \begin{tabular}[c]{@{}l@{}}\completer\\ Version : \completer\end{tabular} \\ \hline
  Développement sous Android      & \begin{tabular}[c]{@{}l@{}}\completer\\ Version : \completer\end{tabular} \\ \hline
  Développement C/ Linux Embarqué & \begin{tabular}[c]{@{}l@{}}\completer\\ Version : \completer\end{tabular} \\ \hline
  Documentation du code C et Java & \begin{tabular}[c]{@{}l@{}}\completer\\ Version : \completer\end{tabular} \\ \hline
  \end{tabular}
  \end{table}

\section{Contrôle des médias}
\subsection{Communiquer entre membres internes du projet}
\completer\\
Pour toute communication interne par mail, les messages devront 
être envoyés en copie à l'adresse suivante :

\subsection{Gestion des médias, sources, références, copyright}
Pour tous les documents utilisés et créés par l'équipe XXX, l'utilisation 
de sources externes au projet est autorisée à la seule condition que chaque 
membre utilisant ces sources ait pris le soin de vérifier les droits 
d'utilisation et de diffusion de cette source ainsi que les normes de 
copyright associées. Dans le cas contraire, la source devra être 
supprimée et remplacée.\\

Pour l'utilisation d'image, schéma ou tout autre document non textuel, 
une légende descriptive est à rajouter en dessous du document inséré.\\


\subsection{Communiquer avec des membres externes au projet}
\subsubsection{Contacter les consultants}
Les consultants devront être contactés en priorité par mail.\\

Les règles relatives au contact des consultants sont disponibles sur le wiki 
ProSE, Onglet : /règles\_pour\_contacter\_client\_et\_consultants/Contacter\_les\_consultants.\\

\completer
Toute la correspondance avec les consultants doit être archivée dans le compte email suivant :\\

Toute prise de rendez-vous de consulting se fait par une demande auprès 
du CdP,  avec autorisation du CdP et reporting auprès du CdP pour garder 
une trace des consulting effectués (pour permettre a minima au CdP de 
tenir à jour la liste des dépenses de consulting. Tous ces consultings 
seront reportés et mis à jour dans le document ....\\

\subsubsection{Contacter le client}
\completer\\
Toute la correspondance mail avec le client doit être archivée dans le compte mail suivant  :\\

Pour garantir une communication optimum entre le client et l'équipe projet, 
seul le chef de projet dispose des droits pour communiquer avec le client. 
Les différents membres de l'équipe devront regrouper les questions pour le 
client lors des réunions d'équipe hebdomadaires pour que le chef de projet 
puisse les transmettre au client (en cas d'urgence, contacter le chef de 
projet directement).\\

De plus, pour chaque écriture d'un mail, le chef de projet demandera à 
l'un de ses RQT une relecture de ce mail avant l'envoi au client.\\

Une règle spécifique de gestion de mails a été mise en place sur ce compte 
mail pour permettre d'archiver simplement et efficacement les échanges 
avec le client et ainsi garder une traçabilité des informations.\\

\section{Contrôle des fournisseurs}
Le cabinet de conseil FORMATO forme et conseille les équipes des projets 
ProSE. Ainsi, certaines formations sont obligatoires et gratuites, tandis 
que d’autres sont payantes et facultatives. De plus, FORMATO propose 
aussi un service consulting sur la plupart des domaines abordés durant 
le projet. La liste des consultants et de leurs spécialités est disponible 
au chapitre Consultants et Auditeurs tandis que les procédures d’inscription 
aux formations et de demandes de consulting sont détaillées ci-après.

\subsection{S'inscrire à une formation}
Toute demande de formation doit être décidée en accord avec le CdP. 
Ce dernier transmettra la demande d'inscription au plus tard 48 heures 
avant le début de la formation.\\
Toutes les demandes effectuées par le chef de projet sont enregistrées 
et archivées dans un document spécifique. Les éléments suivants seront 
à indiquer : la date, le sujet de formation, nom du formateur, durée, 
nombre de membres inscrits, présence des membres inscrits.\\

\subsection{Demander un consulting}
Pour demander un consulting, il est important de prendre un rendez-vous 
même si le consultant est planifié sur la séance ProSE. En dehors des 
séances, les consultants ne sont pas nécessairement disponibles et il 
faut impérativement envoyer une demande par email avec un préavis de 
48h. Il est important de respecter les consignes de rédaction des mails 
(le sujet doit respecter un format particulier. Les règles relatives au 
contact des consultants sont disponibles sur le wiki ProSE, Onglet : 
/règles\_pour\_contacter\_client\_et\_consultants/Contacter\_les\_consultants.\\

\completer

Tout consulting effectué par le chef de projet ou l'équipe est enregistré 
et archivé dans un document spécifique par le chef de projet. Les 
éléments suivants seront à indiquer : la date, le sujet de formation, 
nom du formateur, durée ainsi que les coûts de consulting.\\

Le suivi budget et temps consulting sera également inscrit dans la partie 
"crédit de temps consulting" visible dans le tableau de bord de l'ENTP.\\

\section{Collecte, maintenance et conservation des archives}
\subsection{Le Référentiel Documentaire Projet (RDP)}
Le RDP est le dépôt de tous les artefacts numériques du projet. Il 
est impératif que tous les artefacts numériques produits lors du 
projet y soient stockés sans délais. Ces artefacts seront pris en 
charge par un système de gestion de version, de sorte qu’il sera 
possible de revenir à n’importe quelle version de ces artefacts.

\subsubsection{Structuration du RDP}
\completer
\begin{itemize}
  \item Code : Contient tous les fichiers sources (C et Java), 
  ainsi que les fichiers permettant leur édition, modification, 
  déboggage, test et compilation. Les fichiers objets et 
  exécutables n’y sont pas sauvegardés. Ce répertoire est 
  décomposé en 3 dossiers :
  \begin{itemize}
      \item Exploration : Tous les fichiers utilisés en phase d’exploration. 
      Ces fichiers ne sont pas soumis au respect des 
      conventions de qualités du présent PAQL.
      \item Production : Tous les fichiers permettant la réalisation 
      des logiciels pour le client (toutes les versions)
      \begin{itemize}
          \item Android
          \item Linux
      \end{itemize}
      \item Notice : Contient les notices pour configurer 
      les SVN sur eclipse, jmeter, … etc
  \end{itemize}
  \item Conception
  \begin{itemize}
      \item Ébauches : Ce sont tous les fichiers utilisés au 
      cours de la constitution des dossiers de conception.
      \item Livrables : Documents dès qu’ils ont atteint 
      une première fois l'état « diffusable » ainsi que 
      toutes leurs révisions ultérieures.
      \item Schémas : Toutes les sources des dessins ou 
      fichiers UML produits lors de la phase 
      de conception.
  \end{itemize}
  \item Gestion\_projet
  \begin{itemize}
      \item Client : Tous les documents en rapport avec le client.
      \begin{itemize}
          \item Contrat
          \item Documentation
          \item Mail
      \end{itemize}
      \item Planning : Les plannings du projet (en particulier ceux des audits et des formations)
      \item Réunions : Compte-rendus des réunions.
      \item Audit : Tous les compte-rendus  des différents audits.
  \end{itemize}
  \item Qualite : Cette partie regroupe tous les documents 
  liés à la gestion de la qualité du projet.
  \begin{itemize}
      \item Redaction\_PAQL : Contient tout les documents qualité 
      jugés nécessaires à la production/rédaction du PAQL.
      \item Modèles : Modèle de document et modèles 
      de codes à utiliser pendant le projet.
  \end{itemize}
  \item Spécifications : Même structuration que la conception.
  \item Test : Regroupe tous les artefacts liés à l’activité de 
  test (Cahier de test, Plan Test, Planning).
  \begin{itemize}
      \item Aide\_Instructions
      \item Cahier\_test
      \item Planning
      \item Pland\_test
  \end{itemize}
\end{itemize}

\subsubsection{Obtenir un accès au RDP}
Pour obtenir un accès au RDP, il faut tout d’abord créer un compte 
Redmine sur le site de ProSE : http://prose.eseo.fr/redmine/. Le premier accès 
se fait avec les codes d'accès générique suivant :\\
Login : prose\\
Password : esorp\\
Il  faut ensuite effectuer une création de compte Redmine en suivant 
le lien suivant  : http://prose.eseo.fr/start. Après avoir suivi les 
instructions, il faut ensuite importer une copie du RDP sur sa 
machine (via « svn checkout ... »).\\

\subsubsection{Nom des fichiers des artefacts}
Seules des lettres non accentuées de l'alphabet latin 
(haut de casse, bas de casse et tirets) sont autorisées pour 
le nom du fichier. Le nom du fichier doit se terminer par la 
référence du document (cf. chapitre 3.3, \completer).
\subsubsection{Déposer un artefact sur le RDP}
Pour assurer une traçabilité de qualité, certaines règles sont à 
prendre en compte lors du dépôt d'un quelconque document sur le 
RDP. Le dépôt de document important se fait de façon séparé, et 
en aucun cas le commit simultané de plusieurs fichiers n'est 
autorisé si les modifications ne sont pas identiques ou en lien 
entre elles. De cette façon il sera aisé d'assigner un commentaire 
de « commit » au document déposé, ce commentaire serra annoté d'un 
sigle de la façon suivante :  [SIGLE] :  <Commentaire> :\\

\begin{table}[H]
  \begin{tabular}{|l|l|}
  \hline
  \rowcolor[HTML]{CCCCCC} 
  \textbf{SIGLE} & \textbf{Libellé document}                                           \\ \hline
  CREAT          & Lors de la création du document et de l'ajout sur le RDP            \\ \hline
  MAJ            & Apport d'une nouvelle version d'un document déjà présent sur le RDP \\ \hline
  ATT-REL        & Document en attente de relecture                                    \\ \hline
  ATT-VAL        & Docuement en attente de validation                                  \\ \hline
  VALIDE         & Docuement validé                                                    \\ \hline
  \end{tabular}
\end{table}
\completer

\subsection{Gestion des documents papier}
Afin de garder toutes traces du travail effectué, les documents 
papier seront stockés dans une chemise intitulée : Projet ProSE – 
EquipeXX, chemise conservée par le CdP.\\

A l’intérieur, il sera retrouvé la même classification que le 
premier niveau du RDP. Chaque document papier produit devra 
contenir le nom et prénom de l’auteur du document ainsi que 
la date de création et date de modification.\\

Si cela est matériellement possible, l'auteur du document papier 
pourra faire une copie numérisée du document et la déposer sur 
le RDP pour en garder une trace supplémentaire\\

\section{Formation}
Si des compétences requises par le projet ne sont pas maîtrisées 
par suffisamment de membres dans l’équipe, les membres veilleront 
à acquérir ces compétences par le biais de lectures, de formations 
entre les membres, par des consulting auprès des consultants 
FORMATO ou par des participations à des formations proposées 
par la société FORMATO.\\
\completer

\section{Gestion du risque}
La gestion du risque au sein de notre projet va nous permettre de 
pouvoir anticiper au maximum les risques techniques et humains 
susceptibles de mettre le projet en péril.\\
\completer

\section{Outils et configurations}
\subsection{Réseau informatique}
Le réseau informatique de l’ESEO vous permettra d'accéder à 
internet ainsi qu’à tous les outils cités précédemment. Par 
conséquent, les membres projet sont tenus de respecter la 
charte informatique ESEO (déjà signé lors de l'entrée à l'ESEO).\\

\subsection{Subversion}
\completer

\section{Glossaire : Définitions, acronymes et abréviations}
\completer

\begin{table}[H]
  \begin{tabular}{|l|l|}
  \hline
  \rowcolor[HTML]{CCCCCC} 
  \textbf{Acronyme}                          & \textbf{Définition}                                              \\ \hline
  \ti{AP (Atefact Projet)}                   & \ti{Désigne tous les éléments numériques ou analogiques\\
                                                produits lors du développement du projet\\
                                                (documents numérique ou papier, code sources,\\ 
                                                jeux de test…).}                                                \\ \hline
  \ti{AQ (Assurance Qualité)}                & \ti{\completer}                                                  \\ \hline
  \ti{AuC (Audit Consultatif)}               & \ti{\completer}                                                  \\ \hline
  \ti{AuN (Audit Normatif)}                  & \ti{\completer}                                                  \\ \hline
  \ti{CdP (Chef de Projet)}                  & \ti{Personnage clef du projet, il planifie, dirige, prend\\
                                                   en charge la relation client, le suivi global du\\ 
                                                   projet, l'aspect budgétaire et financier ainsi que \\
                                                   les formations nécessaire à l'équipe}                        \\ \hline
  \ti{ENTP (Espace Numérique \\
      de Travail du Projet).}                & \ti{http://prose.eseo.fr/redmine/projects/se2017\\-equipea2}       \\ \hline
  \ti{IEEE   (Institute of Electrical \\
      and Electronics Engineers)}            & \ti{Association professionnelle internationale définissant\\
                                                   entre autres des normes dans le domaine informatique\\
                                                  et électronique.}                                             \\ \hline
  \ti{OMG (Object Management Group)}         & \ti{Association professionnelle internationale définissant\\
                                                   entre autres des normes dans le domaine informatique.}       \\ \hline
  \ti{RQT (Responsable Qualité et Test)}     & \ti{\completer}                                                  \\ \hline
  \ti{ProSE (Projet Système Embarqué)}       & \ti{\completer}                                                  \\ \hline
  \ti{\completer}                            & \ti{\completer}                                                  \\ \hline
  \ti{PAQL (Plan d’Assurance\\ 
      Qualité Logicielle)}                   & \ti{\completer}                                                  \\ \hline
  \ti{RDP (Référentiel Documentaire\\ 
      Projet) }                              & \ti{Dépôt de tous les artefacts numériques du projet.\\ 
                                                   Ce dépôt est mis à la disposition de l’équipe\\ 
                                                   projet, ainsi qu’à l’équipe des consultants\\
                                                   FORMATO.}                                                    \\ \hline
  \ti{UML (Unified Modeling Language)}       & \ti{Notation graphique normalisée définie par l’OMG et\\
                                                   utilisé en génie logiciel. }                                 \\ \hline
   \end{tabular}
\end{table}

\section{Validation du document}
Signature de tous les membres projets, précédée de la mention 
"J’ai lu et je m’engage à respecter le présent PAQL pendant toute la durée du projet ProSE"

\end{document}



%--- END 

% \end{document}