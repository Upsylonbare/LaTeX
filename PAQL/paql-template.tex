%-----------------------------------
% Plan de test simplifié
%
% ProSE
%
% Auteurs : Thomas CRAVIC
%
%-----------------------------------

\documentclass[a4paper,11pt,titlepage]{article}

\usepackage[english,francais]{babel}
\usepackage[T1]{fontenc}
\usepackage[utf8]{inputenc}

\usepackage{xspace, graphicx}
\usepackage{amsmath}
\usepackage{amsfonts}
\usepackage{amssymb}
\usepackage{lastpage}
\usepackage{array}
\usepackage{tabularx}
\usepackage{hyperref}

\usepackage{titlesec}
\usepackage{float}
\usepackage[table,xcdraw]{xcolor}

%------------------------------------
% Variables
\newcommand{\version}{0.0}
\newcommand{\revision}{0}
\newcommand{\projectName}{Nom du Projet}
\newcommand{\documentName}{Nom du document}
\newcommand{\creator}{Clément Le Goffic}
\newcommand{\creatorAbrev}{C. Le Goffic}
\newcommand{\documentNameAbrev}{XXXX}
\newcommand{\annee}{2022}
\newcommand{\teamName}{Nom d'équipe}
\newcommand{\prose}{ProSE}
\newcommand{\teamNumber}{A?}
\newcommand{\completer}{\textbf{A compléter}}

%------------------------------------

\usepackage{fancyhdr}

\pagestyle{fancy}

\graphicspath{{./figs/}}

\setlength{\hoffset}{-40pt}

\setlength{\topmargin}{-25pt}
\setlength{\headsep}{10pt}

\renewcommand{\headheight}{80pt}

\renewcommand{\headwidth}{450pt}
\setlength{\textwidth}{450pt}
\setlength{\textheight}{604pt}

\renewcommand{\footrulewidth}{0.1mm}

\fancyhf{}
        \fancyhead[LO]{\bf \includegraphics[width=80pt]{eseo.png}\\
        						\medskip
                                 {\prose} équipe {\teamNumber} {\annee}}
         \fancyhead[RO]{\bf \includegraphics[width=40pt]{st.png}\\
				\medskip
				{\small{Ref. {\documentNameAbrev}\_E{\teamNumber}}}}
         \fancyfoot[LO]{\sl {\it Version {\version} - Révision {\revision}}}
         \cfoot{\copyright {\annee} {\teamName}}
         \fancyfoot[RO]{\thepage/{\completer}}

\setcounter{tocdepth}{3}
%%%%%%%%%%%%%%%%%%%%%%%%%%%%%%%%%%%%%%%%%%%%%%%%%%%%%%
\titleclass{\subsubsubsection}{straight}[\subsection]

\newcounter{subsubsubsection}[subsubsection]
\renewcommand\thesubsubsubsection{\thesubsubsection.\arabic{subsubsubsection}}
\renewcommand\theparagraph{\thesubsubsubsection.\arabic{paragraph}} % optional; useful if paragraphs are to be numbered

\titleformat{\subsubsubsection}
  {\normalfont\normalsize\bfseries}{\thesubsubsubsection}{1em}{}
\titlespacing*{\subsubsubsection}
{0pt}{3.25ex plus 1ex minus .2ex}{1.5ex plus .2ex}

\makeatletter
\renewcommand\paragraph{\@startsection{paragraph}{5}{\z@}%
  {3.25ex \@plus1ex \@minus.2ex}%
  {-1em}%
  {\normalfont\normalsize\bfseries}}
\renewcommand\subparagraph{\@startsection{subparagraph}{6}{\parindent}%
  {3.25ex \@plus1ex \@minus .2ex}%
  {-1em}%
  {\normalfont\normalsize\bfseries}}
\def\toclevel@subsubsubsection{4}
\def\toclevel@paragraph{5}
\def\toclevel@paragraph{6}
\def\l@subsubsubsection{\@dottedtocline{4}{7em}{4em}}
\def\l@paragraph{\@dottedtocline{5}{10em}{5em}}
\def\l@subparagraph{\@dottedtocline{6}{14em}{6em}}
\makeatother

\setcounter{secnumdepth}{4}
\setcounter{tocdepth}{4}
%%%%%%%%%%%%%%%%%%%%%%%%%%%%%%%%%%%%%%%%%%%%%
\newcommand{\tabitem}{~~\llap{\textbullet}~~}

%----------------------------------------
%       DOCUMENT
%----------------------------------------

\begin{document}

%---------------------------------------

\sloppy%
% Modifie l’espacement vertical entre les lignes d’un tableau (tabular)
\renewcommand{\arraystretch}{1.5}

%---------------------------------------

%\vspace{-2cm}%
\begin{center}%
\vspace*{1cm}
\rule[0.5ex]{0.4\textwidth}{0.1mm}\\
\vspace*{2mm}
{\Huge {\textsc{\bf {\documentName}}}}
\vspace{0.4cm}\\
{\large\bf {\prose} {\teamNumber} {\annee} - {\teamName}}\\
{\large\bf {\projectName}}

\rule[0.5ex]{0.4\textwidth}{0.1mm}

\vspace{1cm}

\end{center}
\begin{center}
\begin{tabular}{|c|c|}
\hline
Responsable du document & Xxxxxx Xxxxxx                \\
État du document        & En cours de création         \\
Version                 & {\version}  \\
Révision                & {\revision} \\
\hline
\end{tabular}
\end{center}

\vspace{2cm}
\noindent
\textbf{AVERTISSEMENT :}\\
Le présent document est un document à but pédagogique. 
Il a été réalisé sous la direction de Jérôme Delatour, 
en collaboration avec des enseignants 
et des étudiants de l'option SE du groupe ESEO. 
Ce document a été traduit en \LaTeX par Clément Le Goffic.\\
Ce document est la propriété de Jérôme Delatour du groupe ESEO. 
En dehors des activités pédagogiques de l'ESEO, ce document ne peut 
être diffusé ou recopié sans l’autorisation écrite de son propriétaire (Jérôme Delatour). %Comment if not needed

\newpage

% Auteur : Camille Constant

\noindent
% Modifie l’espacement horizontal entre les colonnes
%\setlength{\tabcolsep}{5pt}
\begin{tabularx}{\linewidth}{|p{1.8cm}|X|p{2cm}|p{1.5cm}|p{1.5cm}|}
\hline
\textbf{Date} & \textbf{Actions} & \textbf{Auteur} & \textbf{Version} & \textbf{Révision}  \\
\hline
11/02/2022 & Création du document & {\creatorAbrev} & 0.0 & 0\\
11/02/2022 & Appropriation du document & {\creatorAbrev} & 0.0 & 1\\
\hline
\end{tabularx}

\newpage

%------------------------------- 

\tableofcontents
% alternative pour réduire l'espacement entre les entrées de la table des matières
% (la valeur numérique peut être adaptée au besoin) : 
%{\setlength{\baselineskip}{0.96\baselineskip}\tableofcontents\par}
\newpage

%-------------------------------
% Ajouter toutes les parties les unes après les autres, séparées par un \newpage
% Exemple : 
% \input{Introduction}
% \newpage
\section{But}
\subsection{Objectifs du document}
Ce document est un Plan d’Assurance Qualité Logicielle (PAQL) visant à définir toutes les règles, 
les méthodes et les outils utilisés dans le projet {\projectName} afin de définir et 
contrôler la qualité du projet.\\
Ce document poursuit les objectifs suivants :
\begin{itemize}
  \item Définir le niveau de qualité attendu par l’équipe projet pour le projet ProSE.
  \item Définir les outils utilisés, les processus et procédures à suivre par l’équipe 
  projet tant au niveau organisationnel que technique lors du projet ProSE.
\end{itemize}
Ce document est disponible sur le Référentiel Documentaire Projet(RDP) dans le répertoire 
Qualité\slash Redaction\_PAQL sous le nom "{\documentName}".

\subsection{Portée}
Ce document est destiné à :
\begin{itemize}
  \item À l'équipe projet,
  \item Aux consultants de la société FORMATO
\end{itemize}

\subsection{Copyright}
Le présen document est la propriété de \completer

\subsection{Vue d’ensemble}
Ce PAQL est structuré suivant les grandes parties proposées par la norme [IEEE\-730\_1998].\\
Il est donc décomposé en 24 parties.\\
La norme IEEE 730 décrit le contenu d'un plan d'AQL pour un logiciel:
\begin{itemize}
  \item Intention et Portée
  \item Définitions et Abréviations
  \item Documents de références
  \item Survol du plan d'assurance qualité Logicielle
  \begin{itemize}
    \item Organisation
    \item Niveau de criticité du logiciel
    \item Outils, techniques et méthodologies
    \item Ressources
    \item Normes, pratiques et conventions
    \item Calendriers
  \end{itemize}
  \item Activités et tâches de cycle de vis de l'AQL
  \begin{itemize}
    \item Rôle de l'assurance de produit
    \item Rôle de l'assurance du processus
    \item Assurances sur les activités et les tâches du système de management de la qualité
    \item Activités et tâches additionnelles
  \end{itemize}
  \item Processus et politiques additionnelles
  \begin{itemize}
    \item Processus de revue de contrat
    \item Processus de mesure de la qualité
    \item Politiques de tests
    \item Politique de dérogation et de déviation
    \item Politique d'itération des tâches
  \end{itemize}
  \item Enregistrements et rapports de l'AQL
  \begin{itemize}
    \item Enregistrements
    \item Rapports
  \end{itemize}
\end{itemize}

\subsection{Références}

\nocite{*}
\bibliographystyle{plain}
\bibliography{biblio.bib}

\section{Gestion}
Cette partie décrit l’organisation, les tâches et les responsabilités 
en rapport avec les activités d’Assurance Qualité (AQ) du projet {\projectName}.

\subsection{Organisation}

\subsubsection{Projet concerné}
Le projet \completer


\subsubsection{Ressources humaines}

\subsubsubsection{Équipe Projet}
\noindent
% Modifie l’espacement horizontal entre les colonnes
%\setlength{\tabcolsep}{5pt}
\begin{tabularx}{\linewidth}{|X|X|X|p{5cm}|X|}
\hline
\textbf{Rôle} & \textbf{Nom} & \textbf{Prénom} & \textbf{Mail} & \textbf{Téléphone}  \\
\hline
Responsable Qualité Test & Le Goffic & Clément & clement.legoffic@reseau.eseo.fr & +33 7 82 77 51 25\\
\hline
\end{tabularx}

\subsubsubsection{Client}
\completer

\subsubsubsection{Consultants et auditeurs}
Si nécessaire, l'équipe projet pourra faire appel à la 
société FORMATO en tant que support technique. 
Les consultants et leurs compétences privilégiées sont :
\begin{itemize}
  \item Jérôme DELATOUR (spécification / conception) : jerome.delatour@eseo.fr
  \item Matthias BRUN (codage / tests) : matthias.brun@eseo.fr
  \item Camille CONSTANT (codage/tests) : camille.constant@eseo.fr
\end{itemize}
Des activités d’audits externes (cf. chapitre 5.3, page {\completer}) 
seront menées par les auditeurs de la société FORMATO ou missionnées par elle.

\subsection{Tâches du projet}
\subsubsection{Tâches transversales}
Les tâches transversales de l’assurance qualité incluent les activités suivantes :
\begin{itemize}
  \item Documentation (cf. chapitre 3, page 22).
  \item Revues et audits (cf. chapitre 5 , page 31 )
  \item Inspections internes
  \item Validation et tests
  \item Activités d’amélioration du processus d’AQ
\end{itemize}

\subsubsubsection{Inspections internes}
\completer

\subsubsubsection{Validation et test}
Le Plan de test a pour objectif d’identifier les informations existantes du projet 
et les composants qui doivent être testés. Il énumère les exigences d’évaluation à 
différents niveaux, décrit les stratégies de test qui seront employées, identifie 
les ressources nécessaires et met en évidence les biens livrables pour les tests.\\
\completer
\begin{itemize}
  \item Portée du document, termes et abréviations
  \item Références
  \item Périmètre de test (composants concernés ou non par les tests, 
  fonctionnalités testées ou non, critères d'acceptation des tests)
  \item Processus et stratégie de test(activités, techniques, outils, 
  procédures de test et gestion des anomalies)
  \item Infrastructure de test
  \item Documents de test et livrables
  \item Responsabilités
  \item Équipe de test
  \item Planning prévisionnel
\end{itemize}
Cf : [Plan de test \completer]

\subsubsubsection{Évolution et amélioration du PAQL}
Le PAQL est susceptible d'évoluer au cours du projet, en particulier pour les raisons suivantes :
\begin{itemize}
  \item Toutes les informations nécessaires à la rédaction d'un chapitre 
  ou d'un paragraphe ne sont pas connues ou suffisamment stabilisées lors de la rédaction.
  \item Il s'agit d'une phase du cycle de développement qui sera engagée ultérieurement 
  (cas de la mention « Rédaction réservée »).
  \item Des événements techniques ou organisationnels nécessitant une prise en 
  compte dans le PAQL peuvent apparaître lors du déroulement du projet (modification 
  d'organisation, mise en place de nouvelles normes ou de procédures ou modification 
  de normes ou procédures existantes, ...).
\end{itemize}
Le PAQL est rédigé par le Responsable Qualité et Test (RQT) de l’équipe projet. 
Le Chef de Projet (CdP) et les RQT participent aux décisions de modifications. 
Il incombe au RQT d’effectuer les modifications jugées nécessaires du PAQL. 
En cas de modifications du PAQL, celui-ci devra être signé à nouveau par les 
membres de l’équipe projet.

\subsubsection{Tâches par rapport au cycle de développement}
L’équipe projet suivra un cycle de développement en V en deux incréments. 
Les activités d’AQ sont décrites par rapport à ce cycle. 
Le planning et les échéances associées sont disponibles sur 
l’Espace Numérique de Travail du Projet (ENTP).
\begin{figure}[H]
  \centering
  \includegraphics[width=15cm]{cycle.png}
  \caption{Diagramme du cycle de développement du projet}
\end{figure}

\subsubsubsection{Phase d'initialisation du projet}
\begin{table}[]
  \begin{tabular}{|lll|}
  \hline
  \rowcolor[HTML]{CCCCCC} 
  \multicolumn{3}{|l|}{\cellcolor[HTML]{CCCCCC}\textbf{Phase : Initialisation}}                                                                                                                                                                                                                                                                   \\ \hline
  \multicolumn{3}{|l|}{
    \begin{tabular}[c]{@{}l@{}}
      Objectifs :\\ 
      Prendre en charge le projet, l’organiser, le planifier et en valider les bases.\\ 
      Évaluer les actions nécessaires pour mettre en place le projet.\\ 
      Échanger avec l'équipe sur les règles à définir.
    \end{tabular}}                                                                                                                                                                                                                                 \\ \hline
  \multicolumn{3}{|l|}{
    \begin{tabular}[c]{@{}l@{}}
      Remarques :\\
    \end{tabular}}                                                                                                                                                                                                                                                                \\ \hline
  \multicolumn{1}{|l|}{
    \begin{tabular}[c]{@{}l@{}}
      Acteurs \& reponsabilités :\\ 
      \tabitem CdP et RQT
    \end{tabular}}&\multicolumn{1}{l|}{
    \begin{tabular}[c]{@{}l@{}}
      Méthodes \& Règles :\\ 
      \tabitem Règles pour l'utilisation\\ de l'ENTP\\
      \tabitem Anticipation et organisation\\des deadlines personnelles
    \end{tabular}}& 
    \begin{tabular}[c]{@{}l@{}}
      Moyens \& Outils :\\ 
      \tabitem Initialisation du projet\\ sous ENTP
    \end{tabular}\\ \hline
  \multicolumn{1}{|l|}{
    \begin{tabular}[c]{@{}l@{}}
      Activités \\ d'organisation/pilotage :\\ 
      \tabitem Organisation de la \\réunion de lancement\\ 
      \tabitem Organisation de la phase\\ en aval
    \end{tabular}}& \multicolumn{1}{l|}{
      \begin{tabular}[c]{@{}l@{}}
        Activités de \\ production/soutien :\\ 
        \tabitem Élaboration PAQL\\ 
        \tabitem Mise en place de l’ENTP\\
        \tabitem Définition de la démarche\\ du projet\\
        \tabitem Initialisation du planning \\et du suivi du projet
      \end{tabular}}& 
      \begin{tabular}[c]{@{}l@{}}
        Activités de \\ vérification/contrôle :\\ 
        \tabitem Réunion de lancement\\
        \tabitem \completer
      \end{tabular}\\ \hline
  \multicolumn{1}{|l|}{
    \begin{tabular}[c]{@{}l@{}}
      Produits/données en entrée :\\ 
      \tabitem Wiki Prose et RedMine\\ 
      \tabitem Documents pédagogiques\\
    \end{tabular}}& \multicolumn{1}{l|}{
      \begin{tabular}[c]{@{}l@{}}
        Produits/données en sortie :\\ 
        \tabitem Planning des tâches \\sur l'ENTP\\ 
      \end{tabular}}           & 
      \begin{tabular}[c]{@{}l@{}}
        Produits révisés :\\ 
        \tabitem PAQL
      \end{tabular}\\ \hline
  \multicolumn{3}{|l|}{\begin{tabular}[c]{@{}l@{}}
    Jalons de la phase :\\ 
    \tabitem J1 : \completer
  \end{tabular}}                                                                                                                                                                                                                                                    \\ \hline
  \rowcolor[HTML]{CCCCCC} 
  \multicolumn{1}{|l|}{\cellcolor[HTML]{CCCCCC}
  \begin{tabular}[c]{@{}l@{}}
    Conditions \\ de début de phase :
  \end{tabular}} & \multicolumn{1}{l|}{\cellcolor[HTML]{CCCCCC}
  \begin{tabular}[c]{@{}l@{}}
    Condition \\ de fin de phase :
  \end{tabular}} & 
  \begin{tabular}[c]{@{}l@{}}
    Conditions \\ de passage à la \\phase suivante :
  \end{tabular}\\ \hline
  \multicolumn{1}{|l|}{
    \begin{tabular}[c]{@{}l@{}}
      \tabitem Nomination des CdP et RQT
    \end{tabular}}& \multicolumn{1}{l|}{
      \begin{tabular}[c]{@{}l@{}}
        Truc\\ 
        1
      \end{tabular}}& 
      \begin{tabular}[c]{@{}l@{}}
        Truc5\\ 
        5
      \end{tabular}\\ \hline
  \end{tabular}
\end{table}

\subsection{Responsabilité}
\subsubsection{Définition Générale des rôles}
\subsubsection{Récapitulatif des responsabilités client sur les phases}
\subsubsection{Récapitulatif des responsabilités CdP sur les phases}
\subsubsection{Récapitulatif des responsabilités RQ sur les phases}
\subsubsection{Récapitulatif des responsabilités des développeurs sur les phases}
\section{Documentation}
\subsection{But}
\subsection{Type de documents}
\subsection{Référence des documents}
\subsection{État d’un document}
\subsection{Responsable du document}
\subsection{Processus d’édition d’un document}
\subsection{version d’un document}
\subsection{Format des documents}
\subsubsection{Modèle de document}
\subsubsection{Artefact de code}
\subsubsection{Règles de codage en langage C :}
\subsubsection{Règles de traduction de la conception vers du code en langage C}
\subsubsection{Règles de codage en langage Java}
\subsubsection{Règles de traduction de la conception vers du code en Java}
\subsection{documents internes}
\section{Standards, pratiques, conventions et métriques}
\subsection{But}
\subsection{Exigences qualités générales}
\subsection{Exigences qualités sur les artefacts}
\subsubsection{Exigences sur les documents consultables par les auditeurs}
\subsubsection{Exigences sur les documents livrables}
\subsubsection{Exigences sur le code source}
\section{Revues et Audits}
\subsection{But}
\subsection{Revues}
\subsubsection{Revue de mi-avancement}
\subsubsection{Revue de recette}
\subsection{Audits}
\subsubsection{Audit consultatif}
\subsubsection{Audit normatif}
\subsubsection{Inspection et revue croisée}
\section{Test}
\section{Notification des problèmes et corrections}
\section{Outils, Techniques et Methodologie}
\subsection{L’espace Numérique de Travail du Projet (ENTP)}
\subsubsection{Redmine}
\subsubsection{Planning prévisionnel}
\subsubsection{Suivi du travail}
\subsection{Liste des outils autorisés}
\section{Contrôle des médias}
\subsection{Communiquer entre membres internes du projet}
\subsection{Gestion des médias, sources, références, copyright}
\subsection{Communiquer avec des membres externes au projet}
\subsubsection{Contacter les consultants}
\subsubsection{Contacter le client}
\section{Contrôle des fournisseur}
\subsection{S'inscrire à une formation}
\subsection{Demander un consulting}
\section{Collecte, maintenance et conservation des archives}
\subsection{Le Référentiel Documentaire Projet (RDP)}
\subsubsection{Structuration du RDP}
\subsubsection{Obtenir un accès au RDP}
\subsubsection{Nom des fichiers des artefacts}
\subsubsection{Déposer un artefact sur le RDP}
\subsection{Gestion des documents papier}
\section{Formation}
\section{Gestion du risque}
\section{Outils et configurations}
\subsection{Réseau informatique}
\subsection{Subversion}
\section{Glossaire : Définitions, acronymes et abréviations}
\section{Validation du document}


\end{document}



%--- END 

% \end{document}